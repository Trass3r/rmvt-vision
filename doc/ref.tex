\documentclass{article}
\usepackage{fancybox}
\usepackage{fancyheadings}
%\documentstyle[titlepage,ref,twoside,fancybox]{article}

\newcommand{\unit}[1]{\mbox{$\rm \,#1$}}
\newcommand{\Hz}{\unit{Hz}}
\newcommand{\ms}{\unit{ms}}
\newcommand{\us}{\unit{\mu s}}
\renewcommand{\deg}{${}^\circ$}

\setlength{\topmargin}{0mm}      % top header - top paper
\setlength{\textheight}{230mm} % height of body, not head and footer
\setlength{\textwidth}{128mm}  % width of text
\setlength{\evensidemargin}{36.6mm} % even page - left paper edge to text
\setlength{\oddsidemargin}{36.6mm}  % odd page - left paper edge to text 
\setlength{\headsep}{12mm} %distance bet. header & body text
\setlength{\footskip}{12mm}%distance bet. body of text & baseline of footer
\setlength{\parindent}{0pt} %don't indent paragraphs
\setlength{\parskip}{2mm} %distance between paragraphs
\setlength{\hfuzz}{10pt} %allow for slight overflow to right margin

% Include font file: either NFSS or PostScript:

\def\ds{\displaystyle\strut}
\def\fl{{\lambda}}      % Focal Length


%% Fonts defined with long PostScript names 

\newfont{\Rfon}{NewCenturySchlbk-Roman} % Main text font
\newfont{\School}{NewCenturySchlbk-Roman}
\newfont{\si}{NewCenturySchlbk-Italic}
\newfont{\ital}{NewCenturySchlbk-Italic}
%\newfont{\emph}{NewCenturySchlbk-Italic}
\newfont{\bs}{NewCenturySchlbk-Bold}

% Courier if you have it, otherwise
% substitute Courier: 
\newfont{\vtt}{Courier at 9pt}
\newfont{\Cour}{Courier at 10pt}
\newfont{\Mono}{Courier at 10pt}
\newfont{\Cfon}{Courier at 9pt}  % Computer font
\newfont{\TTfon}{Courier at 9pt}  % Typewriter font for matlab command display
\newfont{\Lfon}{Courier at 7pt}
\newfont{\Zfon}{Courier at 7pt}

\newfont{\Mfon}{Helvetica-Bold at 14pt}
\newfont{\Refon}{Helvetica-Bold at 12pt} % Font for ref pgs headings
\newfont{\Ifon}{NewCenturySchlbk-Bold scaled \magstep2}
\newfont{\Tfon}{NewCenturySchlbk-Bold scaled \magstep4}
\newfont{\Nfon}{NewCenturySchlbk-Bold at 35pt}
\newfont{\NNfon}{Times-Bold at 32pt}
\newfont{\NNNfon}{Times-Bold at 40pt}
\newfont{\NNNNfon}{NewCenturySchlbk-Bold at 40pt}
\newfont{\rbig}{Times-Bold at 27pt} %vphantom in underlining section heads
\newfont{\tms}{Times-Bold at 18pt} %for section heads
\newfont{\tmss}{Helvetica-Bold at 14pt} %for subsection heads
\newfont{\tmsss}{Helvetica-Bold at 12pt} %for subsubsection heads
\newfont{\tmssss}{Helvetica-BoldOblique at 10pt} %for subsubsubsection heads
\newfont{\nineTR}{NewCenturySchlbk-Roman at 9pt}
\newfont{\nineTI}{NewCenturySchlbk-Italic at 9pt}
\newfont{\elevenTI}{NewCenturySchlbk-Italic at 11pt}
\newfont{\bt}{NewCenturySchlbk-Bold at 9pt}
\newfont{\eightTR}{NewCenturySchlbk-Roman at 8pt}
\newfont{\sevenTR}{NewCenturySchlbk-Roman at 7pt}
\newfont{\eightTI}{NewCenturySchlbk-Italic at 8pt}
\newfont{\Heading}{NewCenturySchlbk-Roman at 70pt}
\newfont{\Headingi}{NewCenturySchlbk-Italic at 60pt}
\newfont{\Afon}{NewCenturySchlbk-Roman at 18pt}

\def\und{{\vtt\char"5F}}


% Fonts defined in NFSS 

\newfont{\Rfon}{pncr} % Main text font
\newfont{\School}{pncr}
\newfont{\si}{pncri}
%\newfont{\emph}{pncri}
\newfont{\ital}{pncri}
\newfont{\bs}{pncb}

% MathWorks uses Monospace--this
% NFSS definition is for Courier. 
\newfont{\Cour}{pcrr at 10pt}
\newfont{\Mono}{pcrr at 10pt}
\newfont{\vtt}{pcrr at 9pt}
\newfont{\Cfon}{pcrr at 9pt}  % Computer font
\newfont{\TTfon}{pcrr at 9pt}  % Typewriter font for matlab command display
\newfont{\Lfon}{pcrr at 7pt}
\newfont{\Zfon}{pcrr at 7pt}

\newfont{\Mfon}{phvb at 14pt}
\newfont{\Refon}{phvb at 12pt} % Font for ref pgs headings
\newfont{\Ifon}{pncb scaled \magstep2}
\newfont{\Tfon}{pncb scaled \magstep4}
\newfont{\Nfon}{pncb at 35pt}
\newfont{\NNfon}{ptmb at 32pt}
\newfont{\NNNfon}{ptmb at 40pt}
\newfont{\NNNNfon}{pncb at 40pt}
\newfont{\rbig}{ptmb at 27pt} %just for vphantom in underlining section heads
\newfont{\tms}{ptmb at 18pt} %for section heads
\newfont{\tmss}{phvb at 14pt} %for subsection heads
\newfont{\tmsss}{phvb at 12pt} %for subsubsection heads
\newfont{\tmssss}{phvbo at 10pt} %for subsubsubsection heads
\newfont{\nineTR}{pncr at 9pt}
\newfont{\nineTI}{pncri at 9pt}
\newfont{\elevenTI}{pncri at 11pt}
\newfont{\bt}{pncb at 9pt}
\newfont{\eightTR}{pncr at 8pt}
\newfont{\sevenTR}{pncr at 7pt}
\newfont{\eightTI}{pncri at 8pt}
\newfont{\Heading}{pncb at 70pt}
\newfont{\Headingi}{pncri at 60pt}
\newfont{\Afon}{pncr at 18pt}

\def\und{{\vtt\char"5F}}


% Define reference page commands.
%
%	\mdes	Description
%	\msa	See also
%	\mex	Examples
%	\mlim	Limitations
%	\mdia	Diagnostics
%	\msyn	Synopsis
%	\mpur	Purpose
%	\malg	Algorithm
%	\mcau	Caution
%	\mref	References
%
\newcommand{\mdes}[1]{\hskip -1.25in 
{\hbox{\makebox[1.25in][l]{\Refon Description}}}{#1}\vskip 0.25in}

\newcommand{\msa}[1]{\hskip -1.25in 
{\hbox{\makebox[1.25in][l]{\Refon See Also}}}{#1}\vskip 0.25in}

\newcommand{\mex}[1]{\hskip -1.25in 
{\hbox{\makebox[1.25in][l]{\Refon Examples}}}{#1}\vskip 0.25in}

\newcommand{\mlim}[1]{\hskip -1.25in 
{\hbox{\makebox[1.25in][l]{\Refon Limitations}}}{#1}\vskip 0.25in}

\newcommand{\mdia}[1]{\hskip -1.25in 
{\hbox{\makebox[1.25in][l]{\Refon Diagnostics}}}{#1}\vskip 0.25in}

\newcommand{\msyn}[1]{\hskip -1.25in 
{\hbox{\makebox[1.25in][l]{\Refon Synopsis}}}{\parbox[t]{10cm}{\Cfon #1}}\vskip 0.25in}

\newcommand{\mpur}[1]{\hskip -1.25in 
{\hbox{\makebox[1.25in][l]{\Refon Purpose}}}{#1}\vskip 0.25in}

\newcommand{\malg}[1]{\hskip -1.25in 
{\hbox{\makebox[1.25in][l]{\Refon Algorithm}}}{#1}\vskip 0.25in}

\newcommand{\mcau}[1]{\hskip -1.25in 
{\hbox{\makebox[1.25in][l]{\Refon Cautionary}}}{#1}\vskip 0.25in}

\newcommand{\mref}[1]{\hskip -1.25in 
{\hbox{\makebox[1.25in][l]{\Refon References}}}{#1}\vskip 0.25in}

\newcommand{\mauthor}[1]{\hskip -1.25in 
{\hbox{\makebox[1.25in][l]{\Refon Author}}}{#1}\vskip 0.25in}


\newcommand{\function}[1]{\section*{\kern -1.25in\tms #1\hfill\markboth{#1}{#1}}
\vspace*{-4mm}
\moveleft 1.25in\vbox{\makebox[6.25in][l]{\vrule height .01 in width 6.25 true in}}
\vskip .25 true in}

\newcommand{\vvar}[1]{{\vtt #1}}


\newcommand{\var}[1]{{\vtt #1}}
%\newcommand{\newblock}{}

\newcommand{\namelistlabel}[1]{\mbox{#1}\hfil}
\newenvironment{namelist}[1]{%
\begin{list}{}
{
\let\makelabel\namelistlabel
\settowidth{\labelwidth}{#1}
\setlength{\leftmargin}{1.1\labelwidth}
}
}{%
\end{list}}

\renewcommand{\hom}{homogeneous transform}
\newcommand{\Hom}{Homogeneous transform}
\renewcommand{\vec}[1]{\underline{#1}}
\input{psfig}


\pagestyle{empty}
\begin{document}
\thispagestyle{empty}
\vspace*{\fill}
{\Headingi Machine Vision}

\vspace{2mm}
{\Heading TOOLBOX\\}

\vspace*{2cm}
{\Afon Release 2\ }
\vspace*{2cm}
{\Afon for use with MATLAB}
%% \vspace{1cm}
%% {\Afon Author: Peter I. Corke CSIRO ICT Centre\\
\vspace{4cm}
\hrule
{\Large
\parbox{80mm}{\textbf{Peter I. Corke} \\
%%CSIRO ICT Centre} \hfill {\copyright~CSIRO 2005}
}
%% \hrule
%% \parbox{50mm}{{\Afon Peter I. Corke} \\
%% peter.corke@csiro.au} \hfill {\Afon November 2005}
\newpage
\vspace*{\fill}
\setlength{\fboxsep}{10pt}%
\begin{Bflushleft}[b]
\School Peter I. Corke
%CSIRO\\
%Division of Manufacturing Science and Technology\\
%Queensland Center for Advanced Technology\\
%PO Box 883\\
%Kenmore 4069\\
%1999
\end{Bflushleft}
\vspace*{\fill}

%%\copyright~CSIRO 2005.
%% \noindent
%% %Please note that whilst CSIRO has taken care to ensure that all data included in this material is accurate, no warranties or assurances can be given about 
%% %the accuracy of the contents of this publication.
%% %CSIRO Manufacturing Science and Technolgy makes no warranties, other than 
%% %those required by law, and excludes all liability (including liability for 
%% %negligence) in relation to the opinions, advice or information contained 
%% %in this publication or for any consequences arising from the use of such 
%% %opinion, advice or information.  You should rely on your own independent 
%% %professional advice before acting upon any opinion, advice or information 
%% %contained in this publication.
%% Please note that whilst all care has been taken to ensure that all data included in this material is accurate, no warranties or assurances can be given about 
%% the accuracy of the contents of this publication.
%% Neither CSIRO nor the author makess any warranties, other than 
%% those required by law, and excludes all liability (including liability for 
%% negligence) in relation to the opinions, advice or information contained 
%% in this publication or for any consequences arising from the use of such 
%% opinion, advice or information.  You should rely on your own independent 
%% professional advice before acting upon any opinion, advice or information 
%% contained in this publication.
\noindent
mex- files are based on code which was part of the package VISTA Copyright 1993, 1994 University of British Columbia.
\cleardoublepage

\newcommand{\Mlab}{M\eightTR ATLAB}
\newcommand{\under}[1]{\underline{\rule[-.70ex]{0cm}{4mm}#1}}


{\School

%\pagestyle{headings}        % Gives page headings at top of page
\pagestyle{fancyplain}
\rhead{\rm\thepage}
\lhead{}
\lfoot{Machine Vision Toolbox Release 2}
\rfoot{Copyright (c) Peter Corke 2005}
\cfoot{}
\setlength{\headrulewidth}{0pt}


\newcommand{\be}{\begin{equation}}
\newcommand{\ee}{\end{equation}}
\newcommand{\bea}{\begin{eqnarray}}
\newcommand{\eea}{\end{eqnarray}}
\newcommand{\bi}{\begin{itemize}}
\newcommand{\ei}{\end{itemize}}
\newcommand{\ii}{\item}


\renewcommand{\baselinestretch}{1.2}        % Expands inter-line spacing

\newpage
\rightline{\NNfon Preface}
\vskip 2mm
\moveleft 1.25in\vbox{\makebox[6.25in][l]
{\vrule height .01 in width 6.25 true in}}
\markboth{Introduction}{{}}
\addcontentsline{toc}{section}{Introduction}
\vskip 5mm

\section{Introduction}
The Machine Vision Toolbox (MVT) provides many 
functions that 
are useful in machine vision and vision-based control.
It is a somewhat eclectic collection reflecting the author's personal interest in
areas of photometry, photogrammetry, colorimetry.
It includes over 90 functions spanning operations such as
image file reading and writing, acquisition, display, filtering,
blob, point and line feature extraction,  mathematical morphology, 
homographies, visual Jacobians,
camera calibration and color space conversion.
The Toolbox, combined with Matlab and a modern workstation computer,
is a useful and convenient environment for investigation of machine
vision algorithms.  For modest image sizes the processing rate can
be sufficiently ``real-time'' to allow for closed-loop control.  Focus of
attention methods such as dynamic windowing (not provided) can be
used to increase the processing rate.
With input from a firewire or web camera (support provided) and 
output to a robot
(not provided) it would be possible to implement a visual servo system
entirely in Matlab.

An image is usually treated as a rectangular array of scalar values representing
intensity or perhaps range.
The matrix is the natural datatype for Matlab and thus makes the manipulation
of images easily expressible in terms of arithmetic statements in Matlab
language.
Many image operations such as thresholding, filtering and statistics can
be achieved with existing Matlab functions.
The Toolbox extends this core functionality with M-files that
implement functions and classes, and mex-files for some compute
intensive operations.
It is possible to use mex-files to interface with image acquisition
hardware ranging from simple framegrabbers to robots.
Examples for firewire cameras under Linux are provided.

The routines are written in a straightforward manner which allows
for easy understanding.  Matlab vectorization has been used as much as
possible to improve efficiency, however some algorithms are not amenable
to vectorization.
If you have the Matlab compiler available then this can be used to compile
bottleneck functions.
Some particularly compute intensive functions are provided as mex-files and
may need to be compiled for the particular platform.
This toolbox considers images generally as arrays of double precision
numbers.  This is extravagant on storage, though this is much less
significant today than it was in the past.

This toolbox is not a clone of the Mathwork's own Image Processing 
Toolbox (IPT) although there are many functions in common.
This toolbox predates IPT by many years, is open-source, contains many 
functions that are useful for image feature extraction and control.
It was developed under Unix and Linux systems and some functions
rely on tools and utilities that exist only in that environment.


\subsection{How to obtain the Toolbox}
The Machine Vision Toolbox is available subject to the License
Agreement from the Toolbox home page at
\begin{quote}
http://www.petercorke.com
\end{quote}

The files are available in either gzipped tar format (.gz) or zip
format (.zip).  The web page requests some information from you
regarding such as your country, type of organization and application.
This is just a means for me to gauge interest and to help convince 
myself that this is a worthwhile activity.

\section{Support}
%% There is none!  This software is free to use and you get what you pay for.
%% I'm always happy to correspond with people who have found genuine
No support is provided.  The author is happy to correspond with people who have found genuine
bugs or deficiencies, and 
to accept contributions for inclusion in future versions of the
toolbox, and you will be suitably acknowledged.

I can't guarantee that I respond to your email and I will junk any
requests asking for help with assignments or homework.


\section{Right to use}
Use of the Toolbox is subject to the License Agreement.
Many people are using the Toolbox for teaching and this is something that
the author encourages.  If you plan to duplicate the documentation for class
use then every copy must include the front page.

If you want to cite the Toolbox please use
\begin{verbatim}
@article{Corke05f,
    Author = {P.I. Corke},
    Journal = {IEEE Robotics and Automation Magazine},
    Title = {Machine Vision Toolbox},
    Month = nov,
    Volume = {12},
    Number = {4},
    Year = {2005},
    Pages = {16-25}
}
\end{verbatim}
or
\begin{quote}
"\textit{Machine Vision Toolbox}", P.I. Corke, IEEE Robotics and Automation Magazine, 12(4), pp 16--25, November 2005.
\end{quote}
which is also given in electronic form in the CITATION file.

\subsection{Acknowledgments}
This release includes functions for computing image plane homographies and
the fundamental matrix, contributed by Nuno Alexandre Cid Martins of 
I.S.R., Coimbra.

\section{MATLAB version issues}
The Toolbox works with {\Mlab} version  6 and later.
It has been developed and tested under Suse Linux and Mac OS 10.3.
It has not been tested under Windows.

%\newcommand{\Mlab}{{\School M}{\eightTR ATLAB}}
\newcommand{\MLAB}{{\emph M}{\eightIT ATLAB}}
\newcommand{\undersec}[1]{\underline{\rule[-.70ex]{0cm}{0cm}#1}}


{\School
%\pagestyle{headings}        % Gives page headings at top of page

\newpage   % Starts introduction on RH page
\def\be{\begin{equation}}
\def\ee{\end{equation}}
%\newcommand{\under}[1]{\underline{\rule[-.70ex]{0cm}{0cm}#1}}
\renewcommand{\baselinestretch}{1.2}           % Stretches line spacing
\tolerance=10000


\rightline{\NNNfon 2}
\vskip 4mm
\rightline{\NNfon Reference}
\vskip 2mm
\moveleft 1.25in\vbox{\makebox[6.25in][l]
{\vrule height .01 in width 6.25 true in}}

\vskip 5mm


\addcontentsline{toc}{section}{Introduction}

\markboth{Introduction}{Introduction}

%%SUMMARY_START
%\begin{tabular}{|p{1.25in}p{3.25in}|}
%\cline{1-2} &\\
%\multicolumn{2}{|c|}{\tmsss Reference Page Commands}\\
%\hline
%&\\
%{\Mono ml\_demo1,clc} & User-written function to convert parameter
%vector into a state-space system model.\\
%{\Mono wersum} & Variable augmented with a-priori weighting term.\\
%&\\
%\cline{1-2}
%\end{tabular}

\begin{tabular} {|p{1.25in}p{3.25in}|}
\cline{1-2} &\\
\multicolumn{2}{|c|}{\tmsss Camera modeling and calibration} \\ \hline
{\Mono camcald} & Camera calibration from non-coplanar 3D point data\\
{\Mono camcalp} & Camera calibration from intrinsic and extrinsic parameters\\
{\Mono camcalp\_c} & Camera calibration from intrinsic and extrinsic parameters for
central projection imaging model\\
{\Mono camcalt} & Camera calibration Tsai's method\\
{\Mono camera} & Camera imaging model \\
{\Mono gcamera} & Graphical camera imaging model \\
{\Mono invcamcal} & Inverse camera calibration by Ganapathy's method \\ 
{\Mono pulnix} & Calibration data form Pulnix TN6 camera \\ \hline
\end{tabular}

\begin{tabular}{|p{1.25in}p{3.25in}|}
\cline{1-2} &\\
\multicolumn{2}{|c|}{\tmsss Image plane points and motion} \\ \hline
{\Mono examples/fmtest} & Example of \var{fmatrix()}\\
{\Mono examples/homtest} & Example of \var{homography()}\\
{\Mono epidist} & Distance of points from epipolar lines\\
{\Mono epiline} & Display epipolar lines\\
{\Mono fmatrix}$\ddagger$ & Estimate the fundamental matrix\\
{\Mono frefine}$\ddagger$ & Refine fundamental matrix\\
{\Mono homography}$\ddagger$ & Estimate homography between 2 sets of points\\
{\Mono homtrans} & Transform points by an homography\\
{\Mono invhomog} & Invert an homography\\
{\Mono visjac\_p} & Image Jacobian matrix from points\\ \hline
\end{tabular}

\begin{tabular} {|p{1.25in}p{3.25in}|}
\cline{1-2} &\\
\multicolumn{2}{|c|}{\tmsss Image filtering} \\ \hline
%{\Mono icanny}$\ddagger$ & Canny edge detector\\
{\Mono ismooth} & Gaussian smoothing\\
{\Mono ilaplace} & Laplace filtering\\
{\Mono isobel} & Sobel edge detector\\
{\Mono ipyramid} & Pyramid decomposition\\ 
{\Mono ishrink} & Image smoothing and  shrinking \\ \hline
\end{tabular}

\begin{tabular} {|p{1.25in}p{3.25in}|}
\cline{1-2} &\\
\multicolumn{2}{|c|}{\tmsss Monadic filtering} \\ \hline
{\Mono igamma} & gamma correction\\
{\Mono imono} & convert color to greyscale \\
{\Mono inormhist} & histogram normalization\\ 
{\Mono istretch} & linear normalization\\\hline
\end{tabular}

\begin{tabular} {|p{1.25in}p{3.25in}|}
\cline{1-2} &\\
\multicolumn{2}{|c|}{\tmsss Non-linear filtering} \\ \hline
{\Mono iclose} & greyscale morphological closing\\
{\Mono imorph$\dagger$} & greyscale morphological operations\\
{\Mono iopen} & greyscale morphological opening\\
{\Mono irank$\dagger$} & neighbourhood rank filter\\
{\Mono ivar$\dagger$} & neighbourhood statistics\\
{\Mono iwindow$\dagger$} & generalized neighbourhood operations\\
{\Mono pnmfilt} & Pipe image through Unix utility\\
{\Mono zcross} & zero-crossing detector\\\hline
\end{tabular}

\begin{tabular} {|p{1.25in}p{3.25in}|}
\cline{1-2} &\\
\multicolumn{2}{|c|}{\tmsss Image kernels and structuring elements} \\ \hline
{\Mono kdgauss} & Derivative of 2D Gaussian kernel\\
{\Mono kgauss} & 2D Gaussian kernel\\

{\Mono kdog} & Difference of Gaussians \\
{\Mono klaplace} & Laplacian kernel\\
{\Mono klog} & Laplacian of 2D Gaussian \\
{\Mono kcircle} & Circular mask\\ \hline
\end{tabular}

\begin{tabular} {|p{1.25in}p{3.25in}|}
\cline{1-2} &\\
\multicolumn{2}{|c|}{\tmsss Image segmentation} \\ \hline
{\Mono trainseg} & Return blob features\\
{\Mono colorseg} & Display histogram\\
{\Mono ilabel$\dagger$} & Label an image\\
{\Mono colordistance} & Distance in rg-colorspace\\ 
{\Mono kmeans} & k-means clustering \\
\hline
\end{tabular}

\begin{tabular} {|p{1.25in}p{3.25in}|}
\cline{1-2} &\\
\multicolumn{2}{|c|}{\tmsss Image feature extraction} \\ \hline
{\Mono iblobs} & Return blob features\\
{\Mono ihist} & Display histogram\\
{\Mono ilabel$\dagger$} & Label an image\\
{\Mono imoments} & Compute image moments\\ 
{\Mono iharris} & Harris interest point operator\\ 
{\Mono ihough} & Hough transform (image)\\
{\Mono ihough\_xy} & Hough transform (list of edge points)\\
{\Mono houghoverlay} & overlay Hough line segments\\ \hline
{\Mono houghpeaks} & find peaks in Hough accumulator\\ 
{\Mono houghshow} & show Hough accumulator \\ 
{\Mono max2d} & find largest element in image\\
{\Mono mpq} & compute moments of polygon\\
{\Mono npq} & compute normalized central moments of polygon\\
{\Mono markcorners} & show corner points\\
{\Mono upq} & compute central moments of polygon\\ \hline
\end{tabular}

\begin{tabular} {|p{1.25in}p{3.25in}|}
\cline{1-2} &\\
\multicolumn{2}{|c|}{\tmsss Feature tracking} \\ \hline
{\Mono imatch} & Image template search\\
{\Mono isimilarity} & Image window similarity\\
{\Mono subpixel} & Subpixel interpolation of peak\\
{\Mono zncc} & Region similarity\\
\hline
\end{tabular}

\begin{tabular} {|p{1.25in}p{3.25in}|}
\cline{1-2} &\\
\multicolumn{2}{|c|}{\tmsss Image utilities} \\ \hline
{\Mono idisp} & Interactive image browser \\
{\Mono idisp2} & Non-interactive image browser \\
{\Mono iroi} & Extract region of interest\\
{\Mono xv} & Display image using the XV tool\\ \hline
\end{tabular}

\begin{tabular} {|p{1.25in}p{3.25in}|}
\cline{1-2} &\\
\multicolumn{2}{|c|}{\tmsss Color space/photometry} \\ \hline
{\Mono blackbody} & Blackbody radiation \\
{\Mono ccdresponse} & CCD spectral response \\
{\Mono ccxyz} & CIE XYZ chromaticity coordinate\\
{\Mono cmfxyz} & CIE XYZ color matching function \\
%{\Mono rgb2hsi} & RGB color space to HSI\\
{\Mono rgb2xyz} & RGB color space to CIE XYZ \\
{\Mono rluminos} & relative photopic luminosity (human eye response)\\
{\Mono solar} & solar irradiance spectra \\ \hline
\end{tabular}

\begin{tabular} {|p{1.25in}p{3.25in}|}
\cline{1-2} &\\
\multicolumn{2}{|c|}{\tmsss Image file input/output} \\ \hline
{\Mono firewire}$\dagger$ & read an image from a firewire camera\\
{\Mono loadpgm} & read a PGM format file\\
{\Mono loadppm} & read a PPM format file\\
{\Mono savepnm} & write a PNM format file\\
{\Mono loadinr} & read INRIA INRIMAGE format file\\
{\Mono saveinr} & write INRIA INRIMAGE format file\\ 
%{\Mono video4linux}$\dagger$ & read an image from the Video4Linux driver\\
{\Mono webcam} & read an image from a webcamera\\
{\Mono yuvopen} & open a yuv4mpeg image stream\\
{\Mono yuvread} & read a frame from a yuv4mpeg stream\\
{\Mono yuvr2rgb} & convert a YUV frame to RGB \\
\hline
\end{tabular}

\begin{tabular} {|p{1.25in}p{3.25in}|}
\cline{1-2} &\\
\multicolumn{2}{|c|}{\tmsss Test patterns} \\ \hline
{\Mono lena.pgm} & a famous test image\\
{\Mono mkcube} & return vertices of a cube\\
{\Mono mkcube2} & return edges of a cube\\
{\Mono testpattern} & create range of testpatterns\\
 \hline
\end{tabular}

%%SUMMARY_END
Functions marked with $\ddagger$ are written by others, and their
support of the toolbox is gratefully acknowledged.
Functions marked with $\dagger$ are mex-files and are currently only distributed in
binary form for Linux x86 architecture and as source.
\vfil\eject


%\section*{\kern -1.25in\tms mmle.m\hfill\markboth{mmle.m}{mmle.m}}
%\vspace*{-4mm}
%\moveleft 1.25in\vbox{\makebox[6.25in][l]{\vrule height .01 in width 6.25 true in}}
%\vskip .25 true in
%\mpur{Handle models with known or unknown measurement and/or state noise.}
%
%\synopsis{\Cfon
%{\obeylines
%format compact
%
%load ml\_demo1,clc}
%} 
%
%\mdes{A simple user-written function is created to convert
%the parameter vector into a state-space system model, some input
%options are defined and {\vtt mmle.m} executed.}
%
%\mex{Use standard {\Mlab} \ commands to plot any of the above inputs
%and outputs.
%The innovations can be plotted using :
%\begin{center}
%{\vtt plot(dt$\times$[0:ndp-1], uydata(:,ym)-yest) }
%\end{center}}
%
%\malg{The variable, {\vtt wersum} is
%augmented with the a-priori weighting term to give, using program variable
%names, 
%\be \mbox{wersum}
% =
%\frac{1}{Nm} \left[ \sum_{i=1}^N \mbox{inov}^T \mbox{gg}^{-1} \mbox{inov} + 
%(\mbox{p} - \mbox{pref})^T \mbox{wapriori} (\mbox{p} - \mbox{pref})  \right] 
%\ee
%
%Variable {\vtt inov} is the innovations sequence from the Kalman filter,
%{\vtt gg} is the assumed innovations covariance,
%{\vtt p} is the parameter vector,
%{\vtt pref} is the reference value, and
%{\vtt wapriori} is the a-priori weighting.
%The a-priori weighting can be used to bias the estimated parameters
%in a (possibly wrong) direction favored by the analyst.
%
%The Maximum Likelihood Estimate at the minimum of the LLF  (Log-Likelihood Funct
%ion) is achieved by making
%{\vtt wapriori}  zero and by using the correct value for {\vtt gg}, i.e., \be
%\mbox{gg} \Leftarrow \mbox{rr} \equiv \frac{1}{N} \sum_{i=1}^{N}
%\mbox{\ inov} \mbox{\ inov}^T \label{samcov}
%\ee}
%
%\mcau{Beware of special cases.}
%
%\mlim{Assume a linear time-independent noise model.} 
%
%\msa{\Cfon ml\_p2ss3.m}
%
%\mdia{At each parameter optimization step {\vtt mmle.m} prints :
%\begin{center}
%\begin{tabular}{lcl}
% {\vtt STEP}       &  :& step number \\
%{\vtt PIDF\und nsum}   & :& sum of new {\vtt p(pidf)}, where  {\vtt p} = parameter vector \\
% {\vtt GG\und osum}     & :& old sum of innovations cov. used during the step \\
% {\vtt RR\und nsum}     & :& sum of innovations cov. with new parameters \\
% {\vtt MAXGRAD}    &  :& max(abs(grad)) for {\si old} parameter values \\
% {\vtt MAX$|$dP$|$}   &   :& largest parameter change \\
% {\vtt LLF}      &    :& LLF with new parameter values \\
% {\vtt WERSUM}  &   :& See the {\si Tutorial} and Equation 1. \\
% {\vtt ppid}    &    : & {\vtt p(pid)} = latest identified parameter values \\
%\end{tabular}
%\end{center}}
%
%\mref{Information on references is in the {\si Tutorial}.} 
%
%\vfil\eject 
%

%%%%%%%%%%%%%%%%%%%%%%%%%%%%%%%%%%%%%%%
\function{blackbody}
\mpur{Compute emission spectrum for blackbody radiator}

\msyn{
qdd = blackbody(lambda, T)
}
\mdes{Returns the blackbody radiation in ($\rm W/m^3$) given lambda in (m) and temperature in (K).
If \var{lambda} is a column vector, then \var{E} is a column vector whose elements correspond to
to those in \var{lambda}.}

\mex{To compute the spectrum of a tungsten lamp at 2500K and compare that
with human photopic response.
}
\begin{verbatim}
>> l = [380:10:700]'*1e-9;        % visible spectrum
>> e = blackbody(l, 2500);
>> r = rluminos(l);
>> plot(l*1e9, [e r*1e11])
>> xlabel('Wavelength (nm)')
\end{verbatim}
\noindent
which has the energy concentrated at the red-end (longer
wavelength) of the spectrum.

\psfig{figure=figs/bb-lamp.eps,width=8cm}
\msa{solar}

\vfil\eject 





%%%%%%%%%%%%%%%%%%%%%%%%%%%%%%%%%%%%%%%
\function{camcald}
\mpur{Camera calibration matrix from calibration data}

\msyn{
C = camcald(D)\\[0pt]
[C,resid] = camcald(D);
}

\mdes{\var{camcald} returns a $3 \times 4$ camera calibration
matrix derived from a least squares fit of the data in the matrix
\var{D}.
Each row of \var{D} is of the form \var{[x y z u v]} where
$(x,y,z)$ is the world coordinate of a world point and $(u,v)$ is
the image plane coordinate of the corresponding point.
An optional residual, obtained by back substitution of the calibration data,
can give an indication of the calibration quality.

At least 6 points are required and the points must not be coplanar.
}

\msa{camcalp, camcalt, camera, invcamcal}
\mref{\nocite{Sutherland74}
I.~E. Sutherland, ``Three-dimensional data input by tablet,'' {\em Proc. IEEE},
  vol.~62, pp.~453--461, Apr. 1974.
}
\vfil\eject 

%%%%%%%%%%%%%%%%%%%%%%%%%%%%%%%%%%%%%%%
\function{camcalp}
\mpur{Camera calibration matrix from camera parameters}

\msyn{
C = camcalp(cp, Tcam)\\[0pt]
C = camcalp(cp, pC, x, z)\\[6pt]
C = camcalp\_c(cp, Tcam)\\[0pt]
C = camcalp\_c(cp, pC, x, z)
}

\mdes{Returns a $3 \times 4$ camera calibration
matrix derived from the given camera parameters.
The camera parameter object \var{cp} has elements:

\begin{tabular}{lp{10cm}}
\var{cp.f} & focal length (m) \\
\var{cp.u0} & principal point u-coordinate (pix) \\
\var{cp.v0} & principal point v-coordinate (pix) \\
\var{cp.px} & horizontal pixel pitch (pix/m) \\
\var{cp.py} & vertical pixel pitch (pix/m)
\end{tabular}

The pose of the camera (extrinsic calibration) can be specified 
by the homogeneous transformation
\var{Tcam} or  by specifying the coordinates
of the center, \var{pC}, and unit vectors for the camera's x-axis and 
z-axis (optical axis).

This camera model assumes that the focal point is at $z=0$ and the
image plane is at $z=-f$.  This means that the image is inverted on
the image plane.  Now in a real camera some of these inversions
are undone by the manner in which pixels are rasterized so that
generally increasing X in the world is increasing X on the image plane
and increasing Y in the world (down) is increasing Y on the image
plane.  This has to be handled by setting the sign on the pixel
scale factors to be negative.

\var{camcalp\_c} is a variant for the central projection imaging model,
as opposed to the thin lens model (which includes image inversion).
Such a model is commonly used in computer vision literature where the 
focal point is at $z=0$, and rays pass through the image plane at $z=f$.
This model has no image inversion.
}

\centerline{\psfig{figure=figs/lens.eps,width=6cm}\hfill\psfig{figure=figs/central.eps,width=6cm}}
\centerline{Lens projection model\hfill Central projection model}

\msa{camcald, camcalt, camera, pulnix, invcamcal}
\vfil\eject 


%%%%%%%%%%%%%%%%%%%%%%%%%%%%%%%%%%%%%%%
\function{camcalt}
\mpur{Camera calibration matrix by Tsai's method}

\msyn{
[Tcam,f,k1] = camcalt(D, PAR)
}

\mdes{Returns a $3 \times 4$ camera calibration
matrix derived from from planar calibration data using Tsai's method.
Each row of \var{D} is of the form \var{[x y z u v]} where
$(x,y,z)$ is the world coordinate of a world point and $(u,v)$ is
the image plane coordinate of the corresponding point.
\var{PAR} is a vector of apriori known camera parameters
\var{[Ncx Nfx dx dy Cx Cy]} where \var{Ncx} is 
the number of sensor elements in camera's x direction (in sels),
\var{Nfx} is the number of pixels in frame grabber's x direction (in pixels), 
and \var{(Cx, Cy)} is the 
image plane coordinate of the principal point.

The output is an estimate of the camera's pose, \var{Tcam},
 the focal length, \var{f}, and a lens
 radial distortion coefficient \var{k1}.
}

\mcau{I've never had much luck getting this method to work.  It could be
me, the type of images I take (oblique images are good), or the 
implementation.  The Camera Calibration Toolbox 
\texttt{http://www.vision.caltech.edu/bouguetj/calib\_doc/} gives nice
results.}

\msa{camcalp, camcald, camera, invcamcal}
\mref{\nocite{Tsai86}
R.~Tsai, ``An efficient and accurate camera calibration technique for {3D}
  machine vision,'' in {\em Proc. IEEE Conf. Computer Vision and Pattern
  Recognition}, pp.~364--374, 1986.
}
\vfil\eject 

%%%%%%%%%%%%%%%%%%%%%%%%%%%%%%%%%%%%%%%
\function{camera}
\mpur{Camera projection model}

\msyn{
uv = CAMERA(C, p)\\[0pt]
uv = CAMERA(C, p, Tobj)\\[0pt]
uv = CAMERA(C, p, Tobj, Tcam)
}

\mdes{This function  computes
the transformation from 3D object coordinates to image plane coordinates.
\var{C} is a $3 \times 4$ camera calibration matrix, \var{p} is a matrix
of 3D points, one point per row in X, Y, Z order.
The points are optionally transformed by \var{Tobj}, and the camera is 
optionally transformed by \var{Tcam}, prior to imaging.
The return is a matrix of image plane coordinates, where each row corresponds
to the the row of \var{p}.
}

\mex{Compute the image plane coordinates of a point at $(10,\, 5,\, 30)$
with respect to the standard camera located at the origin.}
\begin{verbatim}
    >> C = camcalp(pulnix)	% create camera calibration matrix
    C =
      1.0e+05 *

      -0.7920         0   -0.3513    0.0027
            0   -1.2050   -0.2692    0.0021
            0         0   -0.0013    0.0000
    >> camera(C, [10 5 30])
    ans =
      479.9736  366.6907

    >> 
\end{verbatim} 


\msa{gcamera, camcalp, camcald}
\vfil\eject 

%%%%%%%%%%%%%%%%%%%%%%%%%%%%%%%%%%%%%%%
\function{ccdresponse}
\mpur{CCD spectral response}

\msyn{
r = ccdresponse(lambda)
}

\mdes{Return a vector of relative response (0 to 1) for a CCD
sensor for the specified wavelength \var{lambda}.
\var{lambda} may be a vector.
}

\psfig{figure=figs/ccd.eps,width=5cm}

\mex{Compare the spectral response of a CCD sensor and the human eye.  We
can see that the CCD sensor is much more sensitive in the red and 
infra-red region than the eye.}
\begin{verbatim}
    >> l = [380:10:700]'*1e-9;
    >> eye = rluminos(l);
    >> ccd = ccdresponse(l);
    >> plot(l*1e9, [eye ccd])
    >> xlabel('Wavelength (nm)')
\end{verbatim}


\mlim{Data is taken from an old Fairchild CCD data book but is somewhat
characteristic of silicon CCD sensors in general.}

\msa{rluminos} 

\vfil\eject 

%%%%%%%%%%%%%%%%%%%%%%%%%%%%%%%%%%%%%%%
\function{ccxyz}
\mpur{
Compute the CIE XYZ chromaticity coordinates}

\msyn{
cc = ccxyz(lambda)\\[0pt]
cc = ccxyz(lambda, e)
}
\mdes{\var{ccxyz} computes the CIE 1931 XYZ chromaticity coordinates
for the wavelength \var{lambda}.
Chromaticity can be computed for an arbitrary spectrum given by the
equal length vectors \var{lambda} and amplitude \var{e}.}

\psfig{figure=figs/locus.eps,width=8cm}

\mex{The chromaticity coordinates of peak green (550\unit{nm}) is}
\begin{verbatim}
>> ccxyz(550e-9)
ans =
    0.3016    0.6924    0.0061
\end{verbatim}
\noindent
and the chromaticity coordinates of a standard tungsten illuminant (color
temperature of 2856\unit{K}) is
\begin{verbatim}
>> lambda = [380:2:700]'*1e-9;        % visible spectrum
>> e = blackbody(lambda, 2856);
>> ccxyz(lambda, e)
ans =
    0.4472    0.4077    0.1451
\end{verbatim}

The spectral locus can be drawn by plotting the chromaticity y-coordinate
against the x-coordinate
\begin{verbatim}
>> xyz = ccxyz(lambda);
>> plot(xyz(:,1), xyz(:,2));
>> xlabel('x'); ylabel('y')
\end{verbatim}

The blackbody locus can be superimposed by
\begin{verbatim}
>> for T=1000:1000:10000,	% from 1,000K to 10,000K
>>   e = blackbody(lambda, T);
>>   xyz = ccxyz(lambda, e);
>>   plot(xyz(1), xyz(2), '*')
>> end
\end{verbatim}
\noindent
which shows points moving from red to white hot (center of the locus) as temperature
increases.


\msa{cmfxyz, blackbody}
\vfil\eject 




%%%%%%%%%%%%%%%%%%%%%%%%%%%%%%%%%%%%%%%
\function{cmfxyz}
\mpur{Color matching function}

\msyn{
xyz = cmfxyz(lambda)
}

\mdes{\var{ccxyz} computes the CIE 1931 color matching functions
for the wavelength \var{lambda} which is returned as a row vector.
If \var{lambda} is a vector then the rows of \var{xyz}
contains the color matching function for the corresponding row of
\var{lambda}.}

\psfig{figure=figs/cmfxyz.eps,width=8cm}
\mex{Plot the X, Y and Z color matching functions as a function
of wavelength.}
\begin{verbatim}
  >> lambda = [350:10:700]'*1e-9;
  >> xyz = cmfxyz(lambda);
  >> for i=1:3,
  >>      subplot(310+i); plot(lambda, xyz(:,i));
  >>   end
\end{verbatim}

\msa{ccxyz}
\vfil\eject 

%%%%%%%%%%%%%%%%%%%%%%%%%%%%%%%%%%%%%%%
\function{colordistance}
\mpur{Distance in rg-colorspace}

\msyn{
r = colordistance(rgb, rg)
}

\mdes{Each pixel of the input color image \var{rgb} is converted to 
normalized $(r,g)$ coordinates 
\begin{eqnarray}
r &=& \frac{R}{R+G+B} \\
g &=& \frac{G}{R+G+B} 
\end{eqnarray}
The Euclidean distance of each pixel from the specificed coordinage \var{rg} is computed
and returned as the corresponding pixel value.

The output is an image with the same number of rows and columns as
\var{rgb} where each pixel represents the correspoding color space distance.

This output image could be thresholded to determine color similarity.
}

\begin{tabular}{cc}
\psfig{figure=figs/targ.eps,width=5cm} &\psfig{figure=figs/colordistance.eps,width=5cm}\\
\end{tabular}

\mex{Show color distance of all targets with respect to a point,
$(200,\, 350)$ on one of the yellow targets}
\begin{verbatim}
    >> targ = loadppm('target.ppm');
    >> pix = squeeze( targ(200,350,:) );
    >> rg = pix / sum(pix);
    >> idisp( colordistance(targ, rg(1:2)), 0.02 )
\end{verbatim}
We use the clipping option of \var{idisp()} to highlight small variations,
since the blue object has a very large color distance.


\msa{colorseg, trainseg} 

\vfil\eject 

%%%%%%%%%%%%%%%%%%%%%%%%%%%%%%%%%%%%%%%
\function{colorseg}
\mpur{Perform rg-space color segmentation}

\msyn{
imseg = colorseg(rgb, map)
}

\mdes{Each pixel of the input color image \var{rgb} is converted to 
normalized $(r,g)$ coordinates 
\begin{eqnarray}
r &=& \frac{R}{R+G+B} \\
g &=& \frac{G}{R+G+B} 
\end{eqnarray}
and these pixels are mapped through the segmentation table \var{map}
to determine whether or not they belong to the desired pixel class.
The map values can be crisp (0 or 1) or fuzzy, though the \var{trainseg()}
creates crisp values.
}

\begin{tabular}{cc}
\psfig{figure=figs/targ.eps,width=5cm} &\psfig{figure=figs/colorseg.eps,width=5cm}\\
\end{tabular}

\mex{Use a pre-trained color segmentation table to segment out the
yellow targets}
\begin{verbatim}
    >> cs = colorseg(targ, map);
    >> idisp(cs);
\end{verbatim}
The segmentation is spotty because the segmentation map is not solid.
We could apply morphological closing to fill the black spots in either the
segmentation map or the resulting segmentation.


\msa{trainseg} 

\vfil\eject 

%%%%%%%%%%%%%%%%%%%%%%%%%%%%%%%%%%%%%%%
\function{epidist}
\mpur{Distance from point to epipolar line}

\msyn{
d = epidist(F, Pa, Pb)
}

\mdes{Given two sets of points \var{Pa} ($n \times 2$) and \var{Pb} (
$m \times 2$ matrix) compute the epipolar line corresponding to each
point in \var{Pa} and the distance of each point in \var{Pb} from each
line.
The result, $d(i,j)$, is a $n \times m$ matrix of distance between
the epipolar line corresponding to $Pa_i$ and the point $Pb_j$.

Can be used to determine point correspondance in a stereo image pair.
}

\msa{fmatrix} 

\vfil\eject 

%%%%%%%%%%%%%%%%%%%%%%%%%%%%%%%%%%%%%%%
\function{epiline}
\mpur{Display epipolar lines}

\msyn{
h = epiline(F, Pa)\\[0pt]
h = epiline(F, Pa, ls)
}

\mdes{Draw epipolar lines in current figure based on points specified
rowwise in \var{Pa} and on the fundamental matrix \var{F}.
Optionally specify the line style \var{ls}.

Adds the lines to the current plot.
}

\mex{Display epipolar lines for the example (\texttt{examples/fmtest}).}
\begin{verbatim}
>> fmtest
 .
 .
>> Fr = frefine(F, uv0, uvf);
>> markfeatures(uvf, 0, '*')
>> epiline(Fr, uv0)
>> grid
\end{verbatim}

\psfig{figure=figs/epilines.eps,width=8cm}

\msa{fmatrix, epidist} 

\vfil\eject 

%%%%%%%%%%%%%%%%%%%%%%%%%%%%%%%%%%%%%%%
\function{firewire}
\mpur{Load an image from a firewire camera}

\msyn{
h = firewire(device, color, framerate)\\[0pt]
im = firewire(h)
}

\mdes{The first form opens the interface and returns a handle or [] on error.
Color is one of 'mono', 'rgb' or 'yuv'.
\var{framerate} is one of the
standard DC1394 rates: 1.875, 3.75, 7.5, 15, 30 or 60 fps.  The
highest rate less than or equal to rate is chosen.

The second form reads an image.  For mono a 2-d matrix is returned,
for rgb a 3-d matrix is returned.  For yuv a structure is
returned with elements \var{.y}, \var{.u} and \var{.v}.

Subsequent calls with the second call format return the next image
from the camera in either grayscale or color format.

}

\mex{Open a firewire camera in rgb mode}.
\begin{verbatim}
>> h = firewire( 0, 'rgb', 7.5);     
CAMERA INFO
===============
Node: 0
CCR_Offset: 15728640x
UID: 0x0814436100003da9
Vendor: Unibrain Model: Fire-i 1.2

>> im = firewire(h);
>> whos im
  Name      Size                   Bytes  Class

  im      480x640x3              7372800  double array

Grand total is 921600 elements using 7372800 bytes

>> 
\end{verbatim}

\mlim{Only \texttt{FORMAT\_VGA\_NONCOMPRESSED} $640 \times 480$ images 
are supported,
and the camera's capabilities are not checked against the requested mode,
for example older Point Grey Dragonflies give weird output when
\var{'mono'} is requested which they don't support.

The achievable frame rate depends on your computer.  The function waits
for the next frame to become available from the camera.  If the function
is called too late you may miss the next frame and have to wait for the
one after that.}

\mlim{Operates only under Linux and is a mex-file.  Requires the
\texttt{libdc1394} and \texttt{libraw1394} libraries to be installed.}

\msa{webcam}
\vfil\eject 

%%%%%%%%%%%%%%%%%%%%%%%%%%%%%%%%%%%%%%%
\function{fmatrix}
\mpur{Estimate the fundamental matrix}

\msyn{
F = fmatrix(Pa, Pb)\\[0pt]
F = fmatrix(Pa, Pb, how)
}

\mdes{
Given two sets of corresponding points \var{Pa} and \var{Pb} (each a
$n \times 2$ matrix)
return the fundamental matrix relating the two sets of observations.

The argument \var{'how'} is used to specify the method and is one of
\var{'eig'}, \var{'svd'}, \var{'pinv'}, \var{'lsq'} (default) or \var{'ransac'}.

RANSAC provides a very robust method of dealing with incorrect
point correspondances through outlier rejection.
It repeatedly uses  one of the underlying methods above in order to 
find inconsistant matches which it then eliminates from the process.
RANSAC mode requires extra arguments:

\begin{tabular}{ll}
\var{iter}    & maximum number of iterations\\
\var{thresh}  & a threshold \\
\var{how} &    the underlying method to use, as above, except for
ransac (optional).
\end{tabular}
Note that the results of RANSAC may vary from run to run due to the
random subsampling performed.

All methods require at least 4 points except \var{'eig'} which requires 
at least 5.  The fundamental matrix is rank 2, ie. \var{det(F)} = 0.
}

\mex{In the following example (\texttt{examples/fmtest}) we will set up a Pulnix camera and a
set of random point features (within a 1m cube) 4m in front of the camera.
Then we will translate
and rotate the camera to get another set of image plane points.  From
the two sets of points we compute the fundamental matrix.}
\begin{verbatim}
>> C=camcalp(pulnix);
>> points = rand(6,3);
>> C = camcalp(pulnix);
>> uv0 = camera(C, points, transl(0,0,4))
uv0 =
  310.7595  293.7777
  367.1380  317.2675
  342.7822  387.1914
  281.4286  323.2531
  285.9791  277.3937
  315.6825  321.7783

>> uvf = camera(C, points, transl(0,0,4), transl(1,1,0)*rotx(0.5))
uvf =
  169.8081  577.5535
  214.7405  579.9254
  207.0585  701.6012
  145.8901  629.0040
  144.5274  559.0554
  154.8456  576.6023

>> F = fmatrix(uv0, uvf)
maximum residual 0.000000 pix

F =
    0.0000   -0.0000    0.0031
    0.0000    0.0000   -0.0027
   -0.0025   -0.0009    1.0000

>> det(F)
ans =
   1.1616e-12
\end{verbatim}
We can see that the matrix is close to singular, theoretically it should
be of rank 2.

\mauthor{Nuno Alexandre Cid Martins, I.S.R.,  Coimbra}

\mref{\nocite{Fischler81,Faugeras93}
M.~A. Fischler and R.~C. Bolles, ``Random sample consensus: a paradigm for
model fitting with applications to image analysis and automated
cartography,'' {\em Communications of the ACM}, vol.~24, pp.~381--395, June
1981.

O.~Faugeras, {\em Three-dimensional computer vision}.
\newblock MIT Press, 1993.
}

\msa{homography, epidist, examples/fmtest}
\vfil\eject 

%%%%%%%%%%%%%%%%%%%%%%%%%%%%%%%%%%%%%%%
\function{frefine}
\mpur{Refine fundamental matrix estimate}

\msyn{
Fr = frefine(F, Pa, Pb)
}

\mdes{
Given two sets of corresponding points \var{Pa} and \var{Pb} (each a
$n \times 2$ matrix) and an estimate of the fundamental
matrix \var{F}, refine the estimate using non-linear optimization
to enforce the rank 2 constraint.
}

\mex{In the following example (\texttt{examples/fmtest}) we will set up a Pulnix camera and a
set of random point features (within a 1m cube) 4m in front of the camera.
Then we will translate
and rotate the camera to get another set of image plane points.  From
the two sets of points we compute the fundamental matrix.}
\begin{verbatim}
>> C=camcalp(pulnix);
>> points = rand(6,3);
>> C = camcalp(pulnix);
>> uv0 = camera(C, points, transl(0,0,4));
>> uvf = camera(C, points, transl(0,0,4), transl(1,1,0)*rotx(0.5));
>> F = fmatrix(uv0, uvf);
maximum residual 0.000000 pix
F =

    0.0000    0.0000    0.0011
   -0.0000    0.0000   -0.0009
   -0.0007   -0.0016    1.0000

>> det(F)
ans =
   1.1616e-12
>> Fr = frefine(F, uv0, uvf)
 .
 .
Fr =

   -0.0000    0.0000   -0.0098
   -0.0000    0.0000    0.0033
    0.0106   -0.0267    1.3938
>> det(Fr)
ans =
   7.8939e-19
\end{verbatim}
We can see that the determinant is much closer to zero.

\msa{homography, epidist, examples/fmtest}
\vfil\eject 

%%%%%%%%%%%%%%%%%%%%%%%%%%%%%%%%%%%%%%%
\function{gcamera}
\mpur{Graphical camera projection model}

\msyn{
hcam = gcamera(name, C ,dims)\\[6pt]
uv = gcamera(hcam, p)\\[0pt]
uv = gcamera(hcam, p, Tobj)\\[0pt]
uv = gcamera(hcam, p, Tobj, Tcam)
}

\mdes{This function creates and graphically displays the image plane of a virtual
camera.

The first function creates a camera display with given name and camera 
calibration matrix.
The size, in pixels of the image plane is given by \var{dims} and is of
the form 
\var{[umin umax vmin vmax]}.
The function returns a camera handle for subsequent function calls.

The second form is used to display a list of 3D points \var{p} in the 
image plane of a previously created camera whose handle is \var{hcam}.
The points are optionally transformed by \var{Tobj}, and the camera is 
optionally transformed by \var{Tcam} prior to imaging.
A single Matlab line object (with point marker style) joins those points.
Successive calls redraw this line providing an animation.

If \var{p} has 6 columns rather than 3, then it is considered to represent 
world line
segments, rather than points.  The first three elements of each row
are the coordinates of the segment start, and the last three elements the coordinates
of the end.
Successive calls redraw the line segments providing an animation.
}

\mlim{Mixing calls in point and line mode give unpredicable results.
}


\mex{Create a virtual Pulnix camera situated at the origin with view axis along the world
Zaxis.
Create a cube of unit side and view it after translating it's centre to $(0,\, 0,\, 5)$.  Note that \var{transl} is a function from the Robotics
Toolbox.}
\begin{verbatim}
      >> C = camcalp(pulnix);
      >> h = gcamera('Pulnix', C, [0 511 0 511]);
      >> c = mkcube2;
      >> gcamera(h, c, transl(0, 0, 5));
\end{verbatim}

\psfig{figure=figs/gcamera.eps,width=8cm}

\msa{\var{camera, camcalp, pulnix, mkcube2} 
}
\vfil\eject 

%%%%%%%%%%%%%%%%%%%%%%%%%%%%%%%%%%%%%%%
\function{homography}
\mpur{Estimate an homography}

\msyn{
H = homography(Pa, Pb)\\[0pt]
H = homography(Pa, Pb, how)
}

\mdes{
Given two sets of corresponding points Pa and Pb (each an nx2 matrix)
return the homography relating the two sets of observations.  The homography
is simply a linear transformation of the initial set of points to the
final set of points.

The argument \var{'how'} is used to specify the method and is one of
\var{'eig'}, \var{'svd'}, \var{'pinv'}, \var{'lsq'} (default) or \var{'ransac'}.

RANSAC provides a very robust method of dealing with incorrect
point correspondances through outlier rejection.
It repeatedly uses  one of the underlying methods above in order to 
find inconsistant matches which it then eliminates from the process.
RANSAC mode requires extra arguments:

\begin{tabular}{ll}
\var{iter}    & maximum number of iterations\\
\var{thresh}  & a threshold \\
\var{how} &    the underlying method to use, as above, except for
ransac (optional).
\end{tabular}
Note that the results of RANSAC may vary from run to run due to the
random subsampling performed.

All methods require at least 4 points except \var{'eig'} which requires 
at least 5.
The homography is only defined for points that are coplanar.
}

\mex{In the following example (\texttt{examples/homtest}) we will set up a Pulnix camera and a
set of planar features 8m in front of the camera.  Then we will translate
and rotate the camera to get another set of image plane points.  From
the two sets of points we compute the homography, and then check it by
back subsitution.}
\begin{verbatim}
>> C=camcalp(pulnix);
>> points = [0 0.3 0; -1 -1 0; -1 1 0; 1 -1 0; 1 1 0];
>> C = camcalp(pulnix);
>> uv0 = camera(C, points, transl(0,0,8))
uv0 =
  274.0000  245.2806
  196.7046   92.3978
  196.7046  327.6022
  351.2954   92.3978
  351.2954  327.6022

>> uvf = camera(C, points, transl(0,0,8), transl(2,1,0)*rotx(0.5))
uvf =
  105.8668  621.9923
   41.5179  455.2694
    9.7312  724.0408
  196.5060  455.2694
  185.9104  724.0408

>> H = homography(uv0, uvf)
H =
    0.9573   -0.1338 -136.3047
   -0.0000    0.7376  366.5758
   -0.0000   -0.0005    1.0000

>> homtrans(H, uv0)-uvf
ans =
   1.0e-09 *

   -0.0876    0.0441
   -0.0473    0.2508
    0.1402   -0.3031
   -0.0290   -0.0356
    0.0715    0.2944
\end{verbatim}
\noindent

\mauthor{Nuno Alexandre Cid Martins, I.S.R.,  Coimbra}
\mref{\nocite{Fischler81,Faugeras93}
M.~A. Fischler and R.~C. Bolles, ``Random sample consensus: a paradigm for
model fitting with applications to image analysis and automated
cartography,'' {\em Communications of the ACM}, vol.~24, pp.~381--395, June
1981.

O.~Faugeras, {\em Three-dimensional computer vision}.
\newblock MIT Press, 1993.
}

\msa{homtrans, examples/homtest, fmatrix}
\vfil\eject 


%%%%%%%%%%%%%%%%%%%%%%%%%%%%%%%%%%%%%%%
\function{homtrans}
\mpur{Transform points by an homography}

\msyn{
ph = homtrans(H, p)
}

\mdes{
Apply the homography H to the image-plane points \var{p}.  \var{p} is an 
$n \times 2$ or $n \times 3$ matrix whose rows correspond to individual 
points non-homogeneous or homogeneous form.

Returns points as \var{ph}, an $n \times 3$ matrix where each row is 
the point coordinate in homogeneous form.
}

\mex{See the example for \var{homography()}.}

\msa{homography, examples/homtest}
\vfil\eject 


%%%%%%%%%%%%%%%%%%%%%%%%%%%%%%%%%%%%%%%
\function{iblobs}
\mpur{Compute image blob features}

\msyn{
F = iblobs(image)\\[0pt]
F = iblobs(image, options, ...)
}

\mdes{Returns a vector of structures
containing feature data and moments upto second order
for each connected (4 or 8 way) region in the  image \var{image}.
The image is first labelled and then features are computed for each
region.

The feature structure is an augmented version of that returned by
\var{imoments} and contains in addition
\var{F(i).minx},
\var{F(i).maxx},
\var{F(i).miny},
\var{F(i).maxy}
and \var{F(i).touch} which is true if the region touches the edge of the image.
\var{F(i).shape} is the ratio of the ellipse axes in the range 0 to 1.

The second form allows various options and blob filters to be applied by specifying
name and value pairs.

\begin{tabular}{lp{6cm}}
\var{'aspect'}, \var{ratio} & specify the pixel aspect ratio (default 1) \\
\var{'connect'}, \var{connectivity} & specficy connectivt (default 4) \\
\var{'touch'}, \var{flag} & only return regions whose \var{touch} status matches \\
\var{'area'}, \var{[amin amax]} & only return regions whose area lies within the
specified bounds \\
\var{'shape'}, \var{[smin smax]} & only return regions whose shape measures lies 
within the specified bounds 
\end{tabular}

Note that to turn one element from a vector of structures into a vector
use the syntax \var{[F.x]}.
}

\mex{Compute the blob features for a test pattern with a grid of 
$5 \times 5$ dots.  26 blobs are found, each of the dots (blobs 2--26), and the 
background (blob 1).}
\begin{verbatim}
  >> im = testpattern('dots', 256, 50, 10);
  >> F = iblobs(im)
  26 blobs in image, 26 after filtering
  F = 
  1x26 struct array with fields:
    area
    x
    y
    a
    b
    theta
    m00
    m01
    m10
    m02
    m20
    m11
    minx
    maxx
    miny
    maxy
    touch
    shape
  >> F(1)
  ans = 
     area: 63511
        x: 128.6116
        y: 128.6116
        a: 147.9966
        b: 147.9857
    theta: -0.7854
      m00: 63511
      m01: 8168251
      m10: 8168251
      m02: 1.3983e+09
      m20: 1.3983e+09
      m11: 1.0505e+09
     minx: 1
     maxx: 256
     miny: 1
     maxy: 256
    touch: 1
    shape: 0.9999

  >> F(2)
  ans = 
     area: 81
        x: 25
        y: 25
        a: 5.0966
        b: 5.0966
    theta: 0
      m00: 81
      m01: 2025
      m10: 2025
      m02: 51151
      m20: 51151
      m11: 50625
     minx: 20
     maxx: 30
     miny: 20
     maxy: 30
    touch: 0
    shape: 1
  >>
  >> idisp(im) 
  >> markfeatures(F, 0, 'b*')
\end{verbatim}
The last two lines overlay the centroids onto the original image.  Note
the centroid of the background object close to the middle dot.

\psfig{figure=figs/iblobs.eps,width=8cm}

\msa{imoments, markfeatures, ilabel}
\vfil\eject 


%%%%%%%%%%%%%%%%%%%%%%%%%%%%%%%%%%%%%%%
\function{icanny}
\mpur{Canny edge operator}

\msyn{
e = canny(im)\\[0pt]
e = canny(im, sigma)\\[0pt]
e = canny(im, sigma, th1)\\[0pt]
e = canny(im, sigma, th1, th0)\\[0pt]
}

\mdes{
Finds the edges in a gray scaled image \var{im} using the Canny
method, and returns an image \var{e} where the edges of \var{im} are marked by 
non-zero intensity values.
This is a more sophisticated edge operator than the Sobel.

The optional argument \var{sigma} is the standard deviation for the Gaussian 
filtering phase. Default is 1 pixel.

\var{th1} is the higher hysteresis threshold.
Default is 0.5 times the strongest edge. Setting \var{th1} to zero will 
avoid the (sometimes time consuming) hysteresis.
\var{th0} is the lower hysteresis threshold and defaults to 0.1 times 
the strongest edge.
}

\psfig{figure=figs/canny.eps,width=5cm}

\begin{verbatim}
  >> lena = loadpgm('lena');
  >> ic = icanny(lena);
\end{verbatim}

\mauthor{Oded Comay, Tel Aviv University}
\mref{J. Canny, "A computational approach to edge detection"
IEEE Transactions on Pattern Analysis and Machine Intelligence,
Volume 8(6), November 1986, pp 679 - 698.}
\msa{isobel, ilaplace}
\vfil\eject 

%%%%%%%%%%%%%%%%%%%%%%%%%%%%%%%%%%%%%%%
\function{iclose}
\mpur{Grey scale morphological opening}

\msyn{
im2 = iclose(im)\\[0pt]
im2 = iclose(im, se)\\[0pt]
im2 = iclose(im, se, N)
}

\mdes{Perfoms a greyscale morphological closing on the image
\var{im} using structuring element \var{se} which defaults to
\var{ones(3,3)}.
The operation comprises \var{N} (default 1) consecutive dilations followed by
\var{N} consectutive erosions.

Square structuring elements can be created conveniently using 
\var{ones(N,N)} and circular structuring elements using \var{kcircle(N)}.
}

\begin{tabular}{cc}
\psfig{figure=figs/colorseg.eps,width=5cm} & \psfig{figure=figs/colorseg_closed.eps,width=5cm} \\
\end{tabular}

\mex{We can use morphological closing to fill in the gaps in an
initial segmentation.}
\begin{verbatim}
  >> idisp(cs)
  >> idisp( iclose(cs, ones(5,5) ) );
\end{verbatim}


\msa{imorph, iopen, kcircle}
\vfil\eject 

%%%%%%%%%%%%%%%%%%%%%%%%%%%%%%%%%%%%%%%
\function{idisp}
\mpur{Interactive image display utility}

\msyn{
idisp(im)\\[0pt]
idisp(im, clip)\\[0pt]
idisp(im, clip, n)\\[6pt]
idisp2(im)
}
\mdes{Displays an image browser in a new figure window 
and allows interactive investigation of pixel values, see Figure.

Buttons are created along the top of the window:
\begin{description}
\item[line]	Prompt for two points in the image and display a cross-section
		in a new figure.  This shows intensity along the line between
		the two points selected.
\item[zoom]	Prompt for two points and rescale the image so that this region
		fills the figure.  The zoomed image may itself be zoomed.
\item[unzoom]	Return image scaling to original settings.
\end{description}

Clicking on a pixel displays its value and coordinate in the top row.
Color images are supported.
        
The second form will limit the displayed greylevels.  If \var{clip} is a
scalar pixels greater than this value are set to \var{clip}.  
If \var{clip} is
a 2-vector then pixels less than \var{clip(1)} are set to \var{clip(1)} and those
greater than \var{clip(2)} are set to \var{clip(2)}.  
\var{clip} can be set to [] for
no clipping.  This option is useful to visualize image content when there
is a very high dynamic range.
The \var{n} argument sets the length of the greyscale color map (default 64).

\var{idisp2} is a non-interactive version, that provides the same
display functionality but has no GUI elements.
}
\psfig{figure=figs/idisp.eps,width=8cm}
\msa{iroi, xv}
\vfil\eject 


%%%%%%%%%%%%%%%%%%%%%%%%%%%%%%%%%%%%%%%
\function{igamma}
\mpur{Image gamma correction}

\msyn{
hn = igamma(image, gamma)\\[0pt]
hn = igamma(image, gamma, maxval)
}

\mdes{Returns a gamma corrected version of \var{image},
in which all pixels
are raised to the power \var{gamma}.
Assumes pixels are in the range 0 to \var{maxval}, default \var{maxval} = 1.
}

\msa{inormhist}
\vfil\eject 


%%%%%%%%%%%%%%%%%%%%%%%%%%%%%%%%%%%%%%%
\function{iharris}
\mpur{Harris interest point detector}

\msyn{
P = iharris\\[6pt]
F = iharris(im)\\[0pt]
F = iharris(im, P)\\[0pt]
[F,rawC] = iharris(im, P)\\[0pt]
}

\mdes{Returns a vector of structures describing the corner features
detected in the image \var{im}.
This is a computationally cheap and robust corner feature detector.
The Harris corner strength measure is
\[
C = \hat{I^2}_x \hat{I^2}_y - \hat{I_{xy}}^2 - k (\hat{I^2}_x + \hat{I^2}_y)^2
\]
and the Noble corner detector is
\[
C = \frac{\hat{I^2}_x \hat{I^2}_y - \hat{I_{xy}}^2}{\hat{I^2}_x + \hat{I^2}_y}
\]
Where $\hat{I^2}_x$ and $ \hat{I^2}_y$ are the smoothed, squared,
directional gradients, and $\hat{I_{xy}}^2$ is the smoothed gradient
product.
For a color image the squared gradients are computed for each plane and then summed.

The feature vector contains structures with elements:
\begin{tabular}{lp{10cm}}
\var{F.x} & x-coordinate of the feature \\
\var{F.y} & y-coordinate of the feature \\
\var{F.c} & corner strength of the feature \\
\var{F.grad} & 3-element vector comprising $[\hat{I^2}_x, \, \hat{I^2}_y, \, \hat{I_{xy}}^2]$ the smoothed gradients at the corner.
\end{tabular}

The gradients can be used as a simple signature of the corner to help
match corners between different views.  A more robust method to match
corners is with cross-correlation of a small surrounding region.

There are many parameters available to control this detector, given
by the second argument \var{P}.  The default value of \var{P} can be
obtained by the first call format.

\begin{tabular}{lp{8cm}}
\var{P.k} & $k$ parameter (default 0.04) \\
\var{P.cmin} & minimum corner strength (default 0) \\
\var{P.cMinThresh} & minimum corner strength as a fraction of maximum
detected corner strength (default 0.01) \\
\var{P.deriv} & x-derivative kernel (default is $\left[ \begin{array}{rrr}
   -1/3 & 0 & 1/3 \\
   -1/3 & 0 & 1/3 \\
   -1/3 & 0 & 1/3 \end{array} \right]$) \\
\var{P.sigma} & $\sigma$ of Gaussian for smoothing step (default 1) \\
\var{P.edgegap} & width of region around edge where features cannot be detected (default 2) \\
\var{P.nfeat} & maximum number of features to detect (default all) \\
\var{P.harris} & Harris (1) or Noble corner detector (default 1) \\
\var{P.tiling} & determine strongest features in a \var{P.tiling} $\times$ 
\var{P.tiling} tiling of the image.   Allows more even feature
distribution (default 1).\\
\var{P.distance} & enforce a separation between features (default 0).
\end{tabular}

Optionally returns the raw corner strength image as \var{rawC}.
}

\begin{tabular}{cc}
\psfig{figure=figs/harris.eps,width=4cm} & \psfig{figure=figs/harris2.eps,width=4cm}\\
\end{tabular}

\mex{Find the corners in the Lena image.  Display a white diamond at the
location of the 20 strongest corners and label them.  Enforce a separation
of 20 pixels between features.}
\begin{verbatim}
  >> lena = loadpgm('lena');
  >> P = iharris;
  >> P.distance = 20;
  >> F = iharris(lena, P);
  12250 minima found (4.7%)
  break after 629 minimas
  >> markfeatures(F, 20, 'wd', {10, 'w'})
  >>
  >> P.tiling = 2;
  >> P.nfeat = 10;
  >> F = iharris(lena, P);
  tile (1,1): 1399 minima found (4.8%), break after 17 minimas 6 added
  tile (1,2): 1356 minima found (4.7%),  10 added
  tile (1,3): 1292 minima found (4.4%),  10 added
  tile (2,1): 1378 minima found (4.7%),  10 added
  tile (2,2): 1307 minima found (4.5%),  10 added
  tile (2,3): 1380 minima found (4.7%),  10 added
  tile (3,1): 1230 minima found (4.2%),  10 added
  tile (3,2): 1467 minima found (5.0%), break after 34 minimas 8 added
  tile (3,3): 1344 minima found (4.6%),  10 added

  >> F
  F = 
  1x84 struct array with fields:
    x
    y
    c
    grad

  >> markfeatures(F, 0, 'wd', {10, 'w'}) 
\end{verbatim}
Note that in two of the tiles not enough corners could be found that
met the criteria of inter-corner separation and corner strength.  The
process yielded only 84 corners, not the 90 requested, however the coverage
of the scene is greatly improved.

\msa{markfeatures, zncc, isimilarity}
\mref{\nocite{Harris88}
C.~G. Harris and M.~J. Stephens, ``A {C}ombined {C}orner and {E}dge
  {D}etector,'' in {\em Proceedings of the Fourth Alvey Vision Conference,
  Manchester}, pp.~147--151, 1988.
}\vfil\eject 

%%%%%%%%%%%%%%%%%%%%%%%%%%%%%%%%%%%%%%%
\function{ihist}
\mpur{Compute intensity histogram (fast)}

\msyn{
ihist(im)\\[0pt]
N = ihist(im)\\[0pt]
[N,X] = ihist(im)
}

\mdes{This function computes the intensity histogram of an image.

The first form plots a graph of the histogram, while the last two forms
simply return the histogram and bin values:
\var{N} is the bin count and \var{X} is the bin number.
}

\psfig{figure=figs/ihist.eps,width=8cm}

\mex{Display the histogram of the Lena image.}
\begin{verbatim}
  >> lena = loadpgm('lena');
  >> ihist(lena)
\end{verbatim}

\mlim{Assumes that the pixels are in the range 0-255 and always computes
256 bins.
Some functions to interpret the histogram to find extrema or fit 
Gaussians would be useful, see \var{fit\_ML\_normal} from Matlab file 
exchange.}

\msa{hist, kmeans}
\vfil\eject 

%%%%%%%%%%%%%%%%%%%%%%%%%%%%%%%%%%%%%%%
\function{ihough}
\mpur{Linear Hough transform}

\msyn{
hp0 = ihough\\[0pt]
H = ihough(edge)\\[0pt]
H = ihough(edge, hp)\\[0pt]
H = ihough\_xy(xyz, drange, ntheta)\\[6pt]
houghshow(H)\\[0pt]
houghpeaks(H, n)\\[0pt]
h = houghoverlay(p, ls)
}

\mdes{Computes the linear Hough transform of the image \var{image}.
Non-zero pixels in the input edge image \var{edges} increment all pixels in the
accumulator that lie on the line
\begin{equation}
        d = y \cos(\theta) + x \sin(\theta)
\end{equation}	
where $\theta$ is the angle the line makes to horizontal axis, and $d$ is
the perpendicular distance between (0,0) and the line.  A horizontal
line has $\theta = 0$, a vertical line has $\theta = \pi/2$ or $-\pi/2$.
The accumulator array has theta across the columns and offset down
the rows.
The Hough accumulator cell is incremented by the
absolute value of the pixel value if it exceeds
\var{params.edgeThresh} times the maximum value found in \var{edges}.
Clipping is applied so that only those points lying within the Hough
accumulator bounds are updated.

An alternative form \var{ihough\_xy()} takes a list of coordinates
rather than an image.  \var{xyz} is either an $n \times 2$ matrix of
$(x,y)$ coordinates, each of which is incremented by 1, or an 
$n \times 3$ matrix where the third column is the amount to incrmement
each cell by.

The returned Hough object \var{H} has the elements:

\begin{tabular}{lp{10cm}}
\var{H.h} & Hough accumulator \\
\var{H.theta} & vector of theta values corresponding to accumulator columns\\
\var{H.d} & vector of offset values corresponding to accumulator rows 
\end{tabular}

Operation can be controlled by means of the parameter object \var{hp}
which has elements:

\begin{tabular}{lp{10cm}}
\var{hp.Nd} & number of bins in the offset direction (default 64) \\
\var{hp.Nth} & number of bins in the theta direction (default 64) \\
\var{hp.edgeThresh} & edge threshold (default 0.10) \\
\var{hp.border} & edge threshold (default 8) \\
\var{hp.houghThresh} & threshold on relative peak strength (default 0.40) \\
\var{hp.radius} & radius of accumulator cells cleared around peak after detection (default 5) \\
\var{hp.interpWidth} &  width of region used for peak interpolation (default 5)
\end{tabular}

Pixels within \var{hp.border} of the edge will not increment, useful to
eliminate spurious edge pixels near image border.

Theta spans the range $-\pi/2$ to $\pi/2$ in \var{hp.Nth} increments. Offset is in the range
1 to the number of rows of \var{edges} with \var{hp.Nd} steps.  
For the \var{ihough\_xy} form the number of theta steps is given by
\var{ntheta} and the offset is given by a vector
\var{drange = [dmin dmax]} or 
\var{drange = [dmin dmax Nd]}.

The default parameter values can be obtained by calling \var{ihough} with
no arguments.

\var{houghshow} displays the Hough accumulator as an image.

\var{houghpeaks} returns the coordinates of \var{n} peaks from the Hough
accumulator.  The highest peak is found, refined to subpixel precision,
then \var{hp.radius} radius around that point is zeroed so as to eliminate
multiple close minima.  The process is repeated for all \var{n} peaks.
\var{p} is an $n \times 3$ matrix where each row is the offset, theta and
relative peak strength (range 0 to 1).
The peak detection loop breaks early if the remaining peak has a relative
strength less than \var{hp.houghThresh}.
The peak is refined by a weighted mean over a $w \times w$ region around
the peak where $w = $ \var{hp.interpWidth}.

\var{houghoverlay} draws the lines corresponding to the rows of \var{p}
onto the current figure using the linestyle \var{ls}.  Optionally returns
a vector of handles \var{h} to the lines drawn.
}

\mex{Find the Hough transform of the edges of a large square, created using
\var{mksq} and a Laplacian edge operator.  The accumulator can be displayed
as an image which shows four bright spots, each corresponding to an edge.
As a surface these appear as high, but quite ragged, peaks.}
\begin{verbatim}
  >> im=testpattern('squares', 256, 256, 128);
  >> edges = isobel(im);
  >> idisp(im);
  >> H = ihough(edges)
  H = 

        h: [64x64 double]
    theta: [64x1 double]
        d: [64x1 double]

  >> houghshow(H);
  >> p=houghpeaks(H, 4)
  p =
  191.2381         0    1.0000
  190.9003    1.5647    1.0000
   69.8095    0.0491    0.6455
   70.1650    1.5239    0.6455

  >> idisp(im);
  >> houghoverlay(p, 'g')
  theta = 0.000000, d = 191.238095
  theta = 1.564731, d = 190.900293
  theta = 0.049087, d = 69.809524
  theta = 1.523942, d = 70.164994

\end{verbatim}

\begin{tabular}{cc}
\psfig{figure=figs/hough1.eps,width=6cm} &\psfig{figure=figs/hough2.eps,width=6cm} \\
Edge image & Hough accumulator \\
\psfig{figure=figs/hough3.eps,width=6cm} & \\
Fitted line segments & \\
\end{tabular}

\vfil\eject


%%%%%%%%%%%%%%%%%%%%%%%%%%%%%%%%%%%%%%%
\function{ilabel}
\mpur{Image labelling (segmentation)}

\msyn{
L = ilabel(I)\\[0pt]
[L, maxlabel] = ilabel(I)\\[0pt]
[L, maxlabel, parents] = ilabel(I)
}
\mdes{Returns an equivalent sized image, \var{L},
in which each pixel is the label of the region of the corresponding
pixel in \var{I}.
A region is a spatially contiguous region
of pixels of the same value.
The particular label assigned has no significance, it is an arbitrary
label.

Optionally the largest label can be returned.  All labels lie
between 1 and \var{maxlabel}, and there are no missing values.
Connectivity is 4-way by default, but 8-way can be selected.

The third form returns an array of region hierarchy information.
The value of \var{parents(i)} is the label of the region that fully
encloses region \var{i}.  The outermost blob(s) will have a parent
value of 0.
}
\mex{Consider the simple binary image}
\begin{verbatim}
>> labeltest
 .
 .
>> a
a =
     0     0     0     0     0     0     0     0     0     0
     0     0     0     0     0     0     0     0     0     0
     0     0     1     0     0     0     0     0     0     0
     0     0     0     0     0     0     0     0     0     0
     0     0     0     0     1     0     0     0     0     0
     0     0     0     0     0     0     0     0     0     0
     0     0     0     0     0     0     0     0     0     0
     0     0     0     0     0     0     0     0     0     0
     0     0     0     0     0     0     0     0     0     0
     0     0     0     0     0     0     0     0     0     0
>> [L,lm,p]=ilabel(a)
L =
     1     1     1     1     1     1     1     1     1     1
     1     1     1     1     1     1     1     1     1     1
     1     1     2     1     1     1     1     1     1     1
     1     1     1     1     1     1     1     1     1     1
     1     1     1     1     3     1     1     1     1     1
     1     1     1     1     1     1     1     1     1     1
     1     1     1     1     1     1     1     1     1     1
     1     1     1     1     1     1     1     1     1     1
     1     1     1     1     1     1     1     1     1     1
     1     1     1     1     1     1     1     1     1     1
lm =
     3
p =
     0
     1
     1
\end{verbatim}
\noindent
which indicates that there are 3 labels or regions.  Region 1, the background
has a parent of 0 (ie. it has no enclosing region).  Regions 2 and 3
are fully enclosed by region 1.

To obtain a binary image of all the pixels in region 2, for example,
\begin{verbatim}
>> L==2
>> L==2
ans =
     0     0     0     0     0     0     0     0     0     0
     0     0     0     0     0     0     0     0     0     0
     0     0     1     0     0     0     0     0     0     0
     0     0     0     0     0     0     0     0     0     0
     0     0     0     0     0     0     0     0     0     0
     0     0     0     0     0     0     0     0     0     0
     0     0     0     0     0     0     0     0     0     0
     0     0     0     0     0     0     0     0     0     0
     0     0     0     0     0     0     0     0     0     0
     0     0     0     0     0     0     0     0     0     0
\end{verbatim}

\begin{tabular}{cc}
\psfig{figure=figs/imraw.eps,width=4cm} & \psfig{figure=figs/imlab.eps,width=4cm}\\
Binary image & Labelled image
\end{tabular}

\msa{imoments, iblobs}
\vfil\eject 

%%%%%%%%%%%%%%%%%%%%%%%%%%%%%%%%%%%%%%%
\function{ilaplace}
\mpur{Laplacian filter}

\msyn{
G = ilaplace(image)
}
\mdes{Convolves all planes of the input image with 
the Laplacian filter
\[
\left[ \begin{array}{rrr}
0 & -1 & 0 \\
-1 & 4 & -1 \\
0 & -1 & 0 \end{array} \right]
\]
}

\psfig{figure=figs/ilaplace.eps,width=5cm}

\mex{Laplace filter the Lena image.}
\begin{verbatim}
 >> lena = loadpgm('lena');
 >> idisp( ilaplace( lena) )
\end{verbatim}

\msa{conv2, klog, ismooth, klaplace}
\vfil\eject 

%%%%%%%%%%%%%%%%%%%%%%%%%%%%%%%%%%%%%%%
\function{imatch}
\mpur{Search for matching region}

\msyn{
[xm, s] = imatch(im1, im2, x, y, w2, search)
}

\mdes{Find the best matchin in \var{im2} for the square region in 
image \var{im1} centered at \var{(x, y)} of half-width \var{w2}.
The search is conducted over the region in \var{im2} centered at
 \var{(x, y)} with bounds \var{search}.  
\var{search = [xmin xmax ymin ymax]} relative to the \var{(x, y)}.
If \var{search} is scalar it searches \var{[-s s -s s]}.

Similarity is computed using the
zero-mean normalized cross-correlation similarity measure
\[
ZNCC(A, B)=\frac{\sum (A_{ij}-\overline{A})(B_{ij}-\overline{B})}{\sqrt{\sum (A_{ij
}-\overline{A})\sum (B_{ij}-\overline{B})}}
\]
where $\overline{A}$ and $\overline{B}$ are the mean over the region
being matched. This measure is invariant to illumination offset.
While computationally complex it yields
a well bounded result, simplifying the decision process.
Result is in the range -1 to 1, with
1 indicating identical pixel patterns.

Returns the best fit \var{xm = [dx dy cc]} where \var{(dx, dy)} are
the coordinates of the best fit with respect to \var{(x, y)} and \var{cc}
is the corresponding cross-correlation score.  Optionally it can return
the cross-correlation score at every point in the search space.  This
correlation surface can be used to interpolate the coordinate of the
peak.
}

\psfig{figure=figs/imatch.eps,width=5cm}

\mex{Search for matching region in the Lena test image.}
\begin{verbatim}
>> lena = loadpgm('lena');
>> [xm,s] = imatch(lena, lena, 200, 200, 7, 7);
>> xm
xm =
         0         0    1.0000
>> idisp(s)
\end{verbatim}
The best match occurs as expected at coordinate $(0, \, 0)$ since the
two images are identical.  The correlation surface is shown above.

\msa{zncc, subpixel}
\vfil\eject 

%%%%%%%%%%%%%%%%%%%%%%%%%%%%%%%%%%%%%%%
\function{imoments}
\mpur{Compute image moments}

\msyn{
F = imoments(image)\\[0pt]
F = imoments(rows, cols)\\[0pt]
F = imoments(rows, cols)
}

\mdes{Returns a structure array
containing moments upto second order
for the non-zero pixels in the binary image \var{image}.  The non-zero
pixels are considered as a single `blob' but no connectivity analysis
is performed.  The actual pixel values are used as pixel weights.
In the second form the row and column coordinates of the
region's pixels can be given instead of an image.

For a binary image the return structre \var{F} contains simple `blob' 
features \var{F.area}, \var{F.x}, \var{F.y}, \var{F.a}, \var{F.b}
and \var{F.theta} where \var{(xc, yc)} is the centroid coordinate,
\var{a} and \var{b}
are axis lengths of the ``equivalent ellipse" and \var{theta}
is the angle of the major ellipse axis to the horizontal axis.  

For a greyscale image \var{area} is actually the sum of the pixel values,
and the centroid is weighted by the pixel values.  This can be useful for
sub-pixel estimation of the centroid of a blob taking into account the edge
pixels which contain components of both foreground and background object.

The structure also contains the
raw moments \var{F.m00}, \var{F.m10}, \var{F.m01}, \var{F.m20},
\var{F.m02}, and \var{F.m11}.
}

\mex{An example is to compute the moments of a particular region label.
First we create a test pattern of an array of large dots. }
\begin{verbatim}
>> image = testpattern('dots', 256, 50, 10);
>> l = ilabel(image);
>> binimage = (l == 3);		% look for region 3
>> imoments(binimage)
ans = 

     area: 81
        x: 75
        y: 25
        a: 5.0966
        b: 5.0966
    theta: 0
      m00: 81
      m01: 2025
      m10: 6075
      m02: 51151
      m20: 456151
      m11: 151875

or

>> [r,c] = find(binimage);
>> imoments(r,c)
     .
     .

\end{verbatim}

\msa{markfeatures, ilabel, mpq}
\vfil\eject 

%%%%%%%%%%%%%%%%%%%%%%%%%%%%%%%%%%%%%%%
\function{imono}
\mpur{Convert color image to greyscale}

\msyn{
im = imono(rgb)
}

\mdes{Returns the greyscale information from the 3-plane RGB image
\var{rgb}.
}

\msa{rgb2hsv}
\vfil\eject 


%%%%%%%%%%%%%%%%%%%%%%%%%%%%%%%%%%%%%%%
\function{imorph}
\mpur{Grey scale morphology}

\msyn{
Im = imorph(I, se, op)\\[0pt]
Im = imorph(I, se, op, edge)
}

\mdes{Perform greyscale morphological filtering on
\var{I} with the structuring element defined by the non-zero elements
of \var{se}.  The supported operations are minimum, maximum or difference
(maximum - minimum) specified by \var{op} values of 
\var{`min'}, 
\var{`max'} and
\var{`diff'} respectively.

Square structuring elements can be created conveniently using \var{ones(N,N)} and 
circular structuring elements using \var{kcircle(N)}.

Edge handling flags control what happens when the processing window
extends beyond the edge of the image.   \var{edge} is either:
\begin{description}
\item[\var{'border'}] (default) the border value is replicated
\item['\var{none'}]   pixels beyond the border are not included in the window
\item['\var{trim'}]   output is not computed for pixels whose window crosses
                 the border, hence the output image is reduced all around
		 by half the window size.
\item['\var{wrap'}]   the image is assumed to wrap around, left to right, top to
bottom.
\end{description}

}

\msa{iopen, iclose, kcircle}
\vfil\eject 

%%%%%%%%%%%%%%%%%%%%%%%%%%%%%%%%%%%%%%%
\function{inormhist}
\mpur{Histogram normalization}

\msyn{
hn = inormhist(image)
}

\mdes{Returns the histogram normalized version of \var{image}.
The grey levels of the output image are spread equally over the range 0 to 255.
This transform is commonly used to enhance contrast in a dark image.}

\begin{tabular}{cc}
\psfig{figure=figs/lena.eps,width=5cm} & \psfig{figure=figs/inormhist.eps,width=5cm} \\
\end{tabular}

\mex{Compare raw and histogram normalized images of Lena.}
\begin{verbatim}
  >> lena = loadpgm('lena');
  >> idisp(lena);
  >> idisp( inormhist(lena) );
\end{verbatim}

\msa{ihist}
\vfil\eject 

%%%%%%%%%%%%%%%%%%%%%%%%%%%%%%%%%%%%%%%
\function{invcamcal}
\mpur{Inverse camera calibration}

\msyn{
[P R K delta] = invcamcal(C)
}

\mdes{\var{invcamcal} estimates the camera extrinsic and intrinsic parameters
from a $3 \times 4$ camera calibration matrix.  \var{P} is a vector of 
estimated camera location, and \var{R} is the estimated rotation matrix.
\var{K} is the estimated scale factors [alphax*f alphay*f] where f is 
camera focal length and alphax and alphay are the pixel pitch in the X and
Y directions.
\var{delta} is an estimate of the `goodness' of the calibration matrix
and is interpretted as the cosine of the angle between the X and Y axes,
and is ideally 0.
}

\msa{camcalp, camcald, camcalt}
\mref{\nocite{Ganapathy84b,Ganapathy84}
S.~Ganapathy, ``Camera location determination problem,'' Technical Memorandum
  11358-841102-20-TM, AT\&T Bell Laboratories, Nov. 1984.

S.~Ganapathy, ``Decomposition of transformation matrices for robot vision,'' in
  {\em Proc. IEEE Int. Conf. Robotics and Automation}, pp.~130--139, 1984.
}
\vfil\eject 

%%%%%%%%%%%%%%%%%%%%%%%%%%%%%%%%%%%%%%%
\function{invhomog}
\mpur{Inverse homography}

\msyn{
s = invhomog(H)
}

\mdes{
Estimates the rotation and translation (upto scale) of the Cartesian motion
corresponding to the given homography H of points in a plane.

There are in general multiple solutions, so the return is a structure
array.  Disambiguating the solutions is up to the user!

The elements of the structure are:

\begin{tabular}{lp{5in}}
R & $3 \times 3$ orthonormal rotation matrix \\
t & vector translation direction \\
n & vector normal of the plane \\
d & distance from the plane (not to scale).
\end{tabular}

($R$, $t$) are the Cartesian transformation from the first camera position to
the second.

}

\mlim{Doesn't seem to work well for cases involving rotation.}
\mcau{Not entirely sure this is correct.  Use with caution.}
\msa{homography}
\mref{\nocite{Faugeras88}
O.~Faugeras and F.~Lustman, ``Motion and structure from motion in a piecewise
  planar environment,'' {\em Int. J. Pattern Recognition and Artificial
    Intelligence}, no.~3, pp.~485--508, 1988.
}
\vfil\eject 


%%%%%%%%%%%%%%%%%%%%%%%%%%%%%%%%%%%%%%%
\function{iopen}
\mpur{Grey scale morphological opening}

\msyn{
im2 = iopen(im)\\[0pt]
im2 = iopen(im, se)\\[0pt]
im2 = iopen(im, se, N)
}

\mdes{Perfoms a greyscale morphological opening on the image
\var{im} using structuring element \var{se} which defaults to
\var{ones(3,3)}.
The operation comprises \var{N} (default 1) consecutive erosions followed by
\var{N} consectutive dilations.

Square structuring elements can be created conveniently using \var{ones(N,N)} and 
circular structuring elements using \var{kcircle(N)}.
}

\begin{tabular}{cc}
\psfig{figure=figs/iopen1.eps,width=5cm} & \psfig{figure=figs/iopen2.eps,width=5cm} \\
\end{tabular}

\mex{Using morphological opening to separate two blobs without changing
their size or shape.}
\begin{verbatim}
      >> idisp(im);
      >> idisp( iopen( im, kcircle(3) ) );
\end{verbatim}

\msa{imorph, iclose, kcircle}
\vfil\eject 

%%%%%%%%%%%%%%%%%%%%%%%%%%%%%%%%%%%%%%%
\function{ipyramid}
\mpur{Pyramid decomposition}

\msyn{
Ip = ipyramid(I)\\[0pt]
Ip = ipyramid(I, sigma)\\[0pt]
Ip = ipyramid(I, sigma, N)
}

\mdes{\var{pyramid} returns a pyramidal
decomposition of the input image \var{I}.
Gaussian smoothing with $\sigma = $ \var{sigma} (default is 1) is applied prior to
each decimation step.  If \var{N} is specified then only \var{N} steps
of the pyramid are computed, else decomposition continues down to a 
$1\times 1$ image.

The result is a cell array of images in reducing size order.
}

\psfig{figure=figs/pyramid.eps,width=8cm}
\mex{Let's place each of the images horizontally adjacent and view
the resulting image.}
\begin{verbatim}
    >> lena = loadpgm('lena');
    >> p = ipyramid(lena, 5);
    >> pi = zeros(512, 992);
    >> w = 1;
    >> for i=1:5,
    >>   [nr,nc] = size(p{i});
    >>   pi(1:nr,w:w+nc-1) = p{i};
    >>   w = w + nc;
    >> end
    >> image(pi)
\end{verbatim}

\msa{ishrink, kgauss}
\vfil\eject 

%%%%%%%%%%%%%%%%%%%%%%%%%%%%%%%%%%%%%%%
\function{irank}
\mpur{Fast neightbourhood rank filter}

\msyn{
Ir = irank(I, order, se)\\[0pt]
Ir = irank(I, order, se, nbins)\\[0pt]
Ir = irank(I, order, se, edge)
}

\mdes{\var{irank()} performs a rank filter over the neighbourhood specified 
by \var{se}.  
The \var{order}'th value in rank (1 is lowest)
becomes the corresponding output pixel value.
A histogram method is used with \var{nbins} (default 256).

Square neighbourhoods can be specified conveniently using \var{ones(N,N)} and 
circular neighbourhoods using \var{kcircle(N)}.

Edge handling flags control what happens when the processing window
extends beyond the edge of the image.   \var{edge} is either:
\begin{description}
\item[\var{'border'}] (default) the border value is replicated
\item['\var{none'}]   pixels beyond the border are not included in the window
\item['\var{trim'}]   output is not computed for pixels whose window crosses
                 the border, hence the output image is reduced all around
		 by half the window size.
\item['\var{wrap'}]   the image is assumed to wrap around, left to right, top to
bottom.
\end{description}


}

\mex{To find the median over a $5 \times 5$ square window.  
After sorting the 25 pixels
in the neighbourhood the median will be given by the 12th in rank.}
\begin{verbatim}
      >> ri = irank(lena, 12, ones(5,5));
      image pixel values: 37.000000 to 226.000000
      >> idisp(ri);    
\end{verbatim}

\msa{kcircle}
\vfil\eject 

%%%%%%%%%%%%%%%%%%%%%%%%%%%%%%%%%%%%%%%
\function{iroi}
\mpur{Select region of interest}

\msyn{
subimage = iroi(image)\\[0pt]
[subimage,corners] = iroi(image)\\[0pt]
subimage = iroi(image, corners)
}

\mdes{
The first two forms display the image and a rubber band box to
allow selection of the region of interest.
Click on the top-left corner then stretch the
box while holding the mouse down.
The selected \var{subimage} is output and optionally the coordinates,
\var{corners} of
the region selected which is of the form [top left; bottom right].
	
The last form uses a previously created region matrix and outputs the
corresponding subimage.  Useful for chopping the same region out of
a different image.  Cropping is applied to all planes of a multiplane image.

Works with color images.
}

\msa{idisp}
\vfil\eject 

%%%%%%%%%%%%%%%%%%%%%%%%%%%%%%%%%%%%%%%
\function{ishrink}

\mpur{Smooth and decmimate an image}

\msyn{
Is = ishrink(I)\\[0pt]
Is = ishrink(I, sigma)\\[0pt]
Is = ishrink(I, sigma, N)
}

\mdes{Return a lower resolution representation of the image \var{I}.
The image is first smoothed by a Gaussian with $\sigma = $ \var{sigma}
and then subsampled by a factor \var{N}.
Default values are \var{sigma = 2} and \var{N = 2}.
}

\begin{tabular}{cc}
\psfig{figure=figs/lena.eps,width=5cm} & \psfig{figure=figs/ishrink.eps,width=5cm} \\
\end{tabular}

\mex{}
\begin{verbatim}
    >> lena = loadpgm('lena');
    >> size(lena)
    ans =
       512   512

    >> s = ishrink(lena, 2, 4);
    >> size(s)
    ans =
       128   128

    >> idisp(s)
\end{verbatim}

\msa{kgauss, ipyramid}
\vfil\eject 

%%%%%%%%%%%%%%%%%%%%%%%%%%%%%%%%%%%%%%%
\function{isimilarity}
\mpur{Zero-mean normalized cross-correlation}

\msyn{
m = isimilarity(im1, im2, c1, c2, w)
}

\mdes{
Compute the similarity
between two equally sized image patches $(2w+1) \times (2w+1)$ centered
at coordinate \var{c1} in image \var{im1}, 
and coordinate \var{c2} in image \var{im2}.

Similarity is computed using the
zero-mean normalized cross-correlation similarity measure
\[
ZNCC(A, B)=\frac{\sum (A_{ij}-\overline{A})(B_{ij}-\overline{B})}{\sqrt{\sum (A_{ij
}-\overline{A})\sum (B_{ij}-\overline{B})}}
\]
where $\overline{A}$ and $\overline{B}$ are the mean over the region
being matched. This measure is invariant to illumination offset.
While computationally complex it yields
a well bounded result, simplifying the decision process.
Result is in the range -1 to 1, with
1 indicating identical pixel patterns.
}
\msa{zncc}
\vfil\eject 

%%%%%%%%%%%%%%%%%%%%%%%%%%%%%%%%%%%%%%%
\function{ismooth}
\mpur{Gaussian filter}

\msyn{
G = ismooth(image, sigma)
}
\mdes{Convolves all planes of the image
with a Gaussian kernel of specified \var{sigma}.
}

\psfig{figure=figs/ismooth.eps,width=5cm}

\mex{Smooth the Lena image.}
\begin{verbatim}
 >> lena = loadpgm('lena');
 >> idisp( ismooth( lena, 4) )
\end{verbatim}

\msa{conv2, kgauss}
\vfil\eject 

%%%%%%%%%%%%%%%%%%%%%%%%%%%%%%%%%%%%%%%
\function{isobel}
\mpur{Sobel filter}

\msyn{
Is = isobel(I)\\[0pt]
Is = isobel(I, Dx)\\[0pt]
[ih,iv] = isobel(I)\\[0pt]
[ih,iv] = isobel(I, Dx)

}

\mdes{Returns a Sobel filtered version of image
\var{I} which is the norm of the vertical and horizontal gradients.
If \var{Dx} is specified this x-derivative kernel is used instead 
of the default:
\[
\left[ \begin{array}{ccc}
      -1  & 0  & 1 \\
      -2  & 0  & 2 \\
      -1  & 0  & 1 \end{array} \right]
\]
With two output arguments specified the function will return the 
vertical and horizontal gradient images.
}

\begin{tabular}{cc}
\psfig{figure=figs/isobel.eps,width=5cm} & \psfig{figure=figs/isobel2.eps,width=5cm} \\
\end{tabular}

\mex{}
\begin{verbatim}
    >> lena = loadpgm('lena');
    >> im = isobel(lena);
    >> idisp(im)
    >> im2 = isobel(lena, kdgauss(2));
    >> idisp(im2)
\end{verbatim}


\mcau{The Sobel operator is a simple edge detector and has the disadvantage
of giving fat double edges.}

\msa{kdgauss}
\vfil\eject 

%%%%%%%%%%%%%%%%%%%%%%%%%%%%%%%%%%%%%%%
\function{istretch}
\mpur{Image linear normalization}


\msyn{
hn = istretch(image)\\[0pt]
hn = istretch(image, newmax)
}

\mdes{Returns a  normalized image in which all pixels lie in the range
0 to 1, or 0 to \var{newmax}.}


\msa{inormhist}
\vfil\eject 


%%%%%%%%%%%%%%%%%%%%%%%%%%%%%%%%%%%%%%%
\function{ivar}
\mpur{Fast neighbourhood variance/kurtosis/skewness}

\msyn{
Im = ivar(I, se, op)\\[0pt]
Im = ivar(I, se, op, edge)
}

\mdes{Computes the specified statistic over the pixel neighbourhood
specified by \var{se} and this
becomes the corresponding output pixel value.
The statistic is specified by \var{op} which is either 
\var{'var'}, \var{'kurt'}, or \var{'skew'}.


Square neighbourhoods can be specified conveniently using \var{ones(N,N)} and 
circular neighbourhoods using \var{kcircle(N)}.

Edge handling flags control what happens when the processing window
extends beyond the edge of the image.   \var{edge} is either:
\begin{description}
\item[\var{'border'}] (default) the border value is replicated
\item['\var{none'}]   pixels beyond the border are not included in the window
\item['\var{trim'}]   output is not computed for pixels whose window crosses
                 the border, hence the output image is reduced all around
		 by half the window size.
\item['\var{wrap'}]   the image is assumed to wrap around, left to right, top to
bottom.
\end{description}

}

\mlim{This  is a very powerful and general facility but it requires that
the MATLAB interpretter is invoked on every pixel, which impacts speed.}

\msa{kcircle}
\vfil\eject 

%%%%%%%%%%%%%%%%%%%%%%%%%%%%%%%%%%%%%%%
\function{iwindow}
\mpur{General function of a neighbourhood}

\msyn{
Im = iwindow(I, se, func)\\[0pt]
Im = iwindow(I, se, func, edge)
}

\mdes{For every pixel in the input image it takes all neighbours for which 
the corresponding element in \var{se} are non-zero.  These are packed into
a vector (in raster order from top left) and passed to the specified
Matlab function.  The return value  becomes the corresponding output
pixel value.

Square neighbourhoods can be specified conveniently using \var{ones(N,N)} and 
circular neighbourhoods using \var{kcircle(N)}.

Edge handling flags control what happens when the processing window
extends beyond the edge of the image.   \var{edge} is either:
\begin{description}
\item[\var{'border'}] (default) the border value is replicated
\item['\var{none'}]   pixels beyond the border are not included in the window
\item['\var{trim'}]   output is not computed for pixels whose window crosses
                 the border, hence the output image is reduced all around
		 by half the window size.
\item['\var{wrap'}]   the image is assumed to wrap around, left to right, top to
bottom.
\end{description}

}

\mex{To compute the mean of an image over an annular window at each point.}
\begin{verbatim}
        >> se = kcircle([5 10]);
        >> out = iwindow(image, se, 'mean');
\end{verbatim}

\mlim{This  is a very powerful and general facility but it requires that
the MATLAB interpretter is invoked on every pixel, which impacts speed.}

\msa{iopen, iclose, kcircle}
\vfil\eject 

%%%%%%%%%%%%%%%%%%%%%%%%%%%%%%%%%%%%%%%
\function{klaplace}
\mpur{Laplacian kernel}

\msyn{
k = klaplace
}
\mdes{Returns the Laplacian kernel
\[
\left[ \begin{array}{ccc}
0 & -1 & 0 \\
-1 & 4 & -1 \\
0 & -1 & 0 \end{array} \right]
\]
}

\mex{}
\begin{verbatim}
  >> klaplace
  ans =
     0    -1     0
    -1     4    -1
     0    -1     0
\end{verbatim}

\msa{conv2, klog, kgauss, ilap}
\vfil\eject 

%%%%%%%%%%%%%%%%%%%%%%%%%%%%%%%%%%%%%%%
\function{kcircle}
\mpur{Create a circular mask}

\msyn{
C = kcircle(r)\\[0pt]
C = kcircle(r, w)
}

\mdes{Returns a circular mask of radius \var{r}.
\var{C} is a $(2r+1) \times (2r+1)$ matrix, or in second case
a $w \times w$ matrix.
Elements are one if on or inside the circle, else zero.

If \var{r} is a 2-element vector then it returns an annulus of ones, and
the two numbers are interpretted as inner and outer radii.

Useful as a circular structuring element for morphological filtering.
}

\mex{To create a circular mask of radius 3}
\begin{verbatim}
    >> kcircle(3)
    ans =
 
     0     0     0     1     0     0     0
     0     1     1     1     1     1     0
     0     1     1     1     1     1     0
     1     1     1     1     1     1     1
     0     1     1     1     1     1     0
     0     1     1     1     1     1     0
     0     0     0     1     0     0     0
\end{verbatim}

\msa{imorph, iopen, iclose}
\vfil\eject 

%%%%%%%%%%%%%%%%%%%%%%%%%%%%%%%%%%%%%%%
\function{kdgauss}
\mpur{Create a 2D derivative of Gaussian filter}

\msyn{
G = kgauss(sigma)\\[0pt]
G = kgauss(sigma, w)
}

\mdes{Returns a $(2w+1) \times (2w+1)$ matrix
containing the x-derivative of the 2D Gaussian function 
\[
g(x,y) = -\frac{x}{2 \pi \sigma^2} e^{\frac{x^2+y^2}{2 \sigma^2}}
\]
symmetric about the center pixel of the matrix.
This kernel is useful for computing smoothed deriviatives.  The 
y-derivative of the Gaussian is simply the transform of this function.

Standard deviation is \var{sigma}.  If \var{w} is not specified 
it defaults to $2\sigma$.

This kernel can be used as an edge detector and is sensitive to
edges in the x-direction.
}

\mex{}
\begin{verbatim}
    >> g = kdgauss(2, 5);
    >> surfl([-5:5], [-5:5], g);
\end{verbatim}

\psfig{figure=figs/kdgauss.eps,width=8cm}

\msa{conv2}
\vfil\eject 

%%%%%%%%%%%%%%%%%%%%%%%%%%%%%%%%%%%%%%%
\function{kdog}
\mpur{Create a 2D difference of Gaussian filter}

\msyn{
LG = kdog(sigma1, sigma2)\\[0pt]
LG = kdog(sigma1, sigma2, w)
}
\mdes{Returns a $(2w+1) \times (2w+1)$ matrix
containing the difference of two 2-D Gaussian functions.
\[
{DoG}(x,y) = \frac{1}{2\pi} \left( e^{-\frac{x^2+y^2}{2 \sigma_1^2}} 
- e^{-\frac{x^2+y^2}{2 \sigma_2^2}} \right)
\]
The kernel is
symmetric about the center pixel of the matrix.
If \var{w} is not specified it defaults to twice the largest $\sigma$.

This kernel can be used as an edge detector and is sensitive to
edges in any direction.
}

\mex{}
\begin{verbatim}
    >> dog = kdog(2, 1.5, 5);
    >> surfl([-5:5], [-5:5], dog);
\end{verbatim}

\psfig{figure=figs/kdog.eps,width=8cm} 

%\begin{tabular}{cc}
%\psfig{figure=figs/ilog.eps,width=5cm} & \psfig{figure=figs/loglena.eps,width=5cm} \\
%\end{tabular}

\msa{conv2, kgauss, klaplace, klog}
\vfil\eject 


%%%%%%%%%%%%%%%%%%%%%%%%%%%%%%%%%%%%%%%
\function{kgauss}
\mpur{Create a 2D Gaussian filter}

\msyn{
G = kgauss(sigma)\\[0pt]
G = kgauss(sigma, w)
}

\mdes{Returns a $(2w+1) \times (2w+1)$ matrix
containing a 2-D Gaussian function 
\[
G(x,y) = \frac{1}{2\pi} e^{-\frac{x^2+y^2}{2 \sigma^2}}
\]
symmetric about the center pixel of the matrix.
The volume under the curve is unity.

Standard deviation is \var{sigma}.  If \var{w} is not specified 
it defaults to $2\sigma$.
}

\mex{}
\begin{verbatim}
    >> g = kgauss(2, 5);
    >> surfl([-5:5], [-5:5], g);
\end{verbatim}

\psfig{figure=figs/kgauss.eps,width=8cm}

\msa{conv2}
\vfil\eject 

%%%%%%%%%%%%%%%%%%%%%%%%%%%%%%%%%%%%%%%
\function{klog}
\mpur{Create a 2D Laplacian of Gaussian filter}

\msyn{
LG = klog(sigma)
}
\mdes{Returns a $(2w+1) \times (2w+1)$ matrix
containing the Laplacian of the 2-D Gaussian function.
\[
{LoG}(x,y) = \frac{-1}{2\pi\sigma^4} (2 - \frac{x^2+y^2}{\sigma^2}) e^{-\frac{x^2+y^2}{2 \sigma^2}}
\]
The kernel is
symmetric about the center pixel of the matrix.
Standard deviation is \var{sigma}.  If \var{w} is not specified 
it defaults to $2\sigma$.

This kernel can be used as an edge detector and is sensitive to
edges in any direction.
}

\mex{}
\begin{verbatim}
    >> lg = klog(2, 5);
    >> surfl([-5:5], [-5:5], lg);
    >> lena = loadpgm('lena');
    >> loglena = conv2(lena, klog(2), 'same');
    >> image(abs(zl))
    >> colormap(gray(15))
\end{verbatim}


\begin{tabular}{cc}
\psfig{figure=figs/ilog.eps,width=5cm} & \psfig{figure=figs/loglena.eps,width=5cm} \\
\end{tabular}

\msa{conv2, kgauss, klaplace, ilap}
\vfil\eject 

%%%%%%%%%%%%%%%%%%%%%%%%%%%%%%%%%%%%%%%
\function{kmeans}
\mpur{k-means clustering}

\msyn{
[c,s] = kmeans(x, N)\\[0pt]
[c,s] = kmeans(x, N, x0)
}

\mdes{Find \var{N} clusters for the data \var{x}.
Returns \var{c} the centers of each cluster as well as \var{s}
which contains the cluster index for each corresponding element of \var{x}.

The initial cluster centers are uniformly spread over the range of
\var{x} but can be specified by an optional N-element vector \var{x0}.

The clustering is performed only with respect to data values, not
spatially.

}

\mex{Can be used for image segmentation, to find pixels with similar
greyscale or hue values.
Segment the Lena image into 4 greyscale bands:}
\begin{verbatim}
    >> [c,s] = kmeans(lena, 4);
    >> c
    c =
       63.7419  108.5058  143.6484  183.0839
    >> idisp(s);
\end{verbatim}

\psfig{figure=figs/kmeans.eps,width=8cm}
The pixels have been clustered into 4 groups with the center values shown.

\mlim{This is an iterative algorithm which is very slow as an m-file.}
\mref{Tou and Gonzalez, Pattern Recognition Principles, pp 94}

\vfil\eject 


%%%%%%%%%%%%%%%%%%%%%%%%%%%%%%%%%%%%%%%
\function{loadinr}
\mpur{Load INRIMAGE format image}

\msyn{
I = loadinr(fname)
}

\mdes{Returns a matrix containing a gray scale
image read from an INRIMAGE format file with the 
specified name.  If no extension is
provided an extension of \texttt{.inr} is appended.
This is a binary floating point file format developed at INRIA.
Returns \var{[]} if the file cannot be opened.

}
\mlim{Only simple 2D images are supported in this implementation.}

\msa{saveinr}
\vfil\eject 

%%%%%%%%%%%%%%%%%%%%%%%%%%%%%%%%%%%%%%%
\function{loadpgm}
\mpur{Load PGM format image (P2 or P5)}

\msyn{
I = loadpgm(fname)
}

\mdes{Returns a matrix containing the gray scale
image read from the specified file.  If no extension is
provided an extension of \texttt{.pgm} is appended.

The given \var{fname} is globbed and if it matches more than 1 file then
the files are read sequentially and a 3-dimensional array is returned where
the last index is the frame number.

If no file is given then a GUI file browser is popped up.

The header parsing is fairly complete and allows for embedded comments
which complicate what would otherwise be a simple header to read.
Returns \var{[]} if the file cannot be opened.
}

\mex{To compute the mean of an image over an annular window at each point.}
\begin{verbatim}
        >> lena = loadpgm('lena');
        >> idisp(lena);
\end{verbatim}

\mlim{Currently does not return the comment field from the file header.}
\msa{savepnm}
\vfil\eject 

%%%%%%%%%%%%%%%%%%%%%%%%%%%%%%%%%%%%%%%
\function{loadppm}
\mpur{Load PPM format image (P3 or P6)}

\msyn{
rgb = loadppm(fname)
}

\mdes{Returns a 3-dimensional matrix containing the red,
green, and blue planes of the 
image read from the specified file.  If no extension is
provided an extension of \texttt{.ppm} is appended.

The given \var{fname} is globbed and if it matches more than 1 file then
the files are read sequentially and a 4-dimensional array is returned where
the last index is the frame number.

If no file is given then a GUI file browser is popped up.

The header parsing is fairly complete and allows for embedded comments
which complicate what would otherwise be a simple header to read.
Returns \var{[]} if the file cannot be opened.
}

\mlim{Currently does not return the comment field from the file header.}

\msa{savepnm, loadpgm, loadinr}
\vfil\eject 

%%%%%%%%%%%%%%%%%%%%%%%%%%%%%%%%%%%%%%%
\function{markfeatures}
\mpur{Mark features}

\msyn{
markfeatures(xy)\\[0pt]
markfeatures(xy, N)\\[0pt]
markfeatures(xy, N, marker)\\[0pt]
markfeatures(xy, N, marker, label)\\[0pt]
}

\mdes{Mark features on the current figure.
The features are specified by \var{xy} which can be an $n \times 2$
matrix, with one row per feature, or a structure vector where each element
has a \var{x} and \var{y} element.
The second form limits the display to at most \var{N} features, if 
\var{N} is zero, then all features are displayed.

The third form allows the marker to be specified with standard Matlab
linestyle specifiers to indicate shape and color.

The fourth form causes the features to be numbered.  \var{label} is a 
2-element cell array where \var{label{1}} is the font size and
\var{label{2}} is the color.
}

\mlim{The feature labelling should better position the label.}

\mex{See the example for \var{iharris}.}

\msa{iharris}
\vfil\eject 

%%%%%%%%%%%%%%%%%%%%%%%%%%%%%%%%%%%%%%%
\function{max2d}
\mpur{Find maximum point in image}

\msyn{
[r,c] = max2d(image)
}

\mdes{Return the interpolated coordinates (r,c) of the greatest peak in image.
Useful for finding peaks in a Hough transform accumulator.}
\msa{ihough}
\vfil\eject 

%%%%%%%%%%%%%%%%%%%%%%%%%%%%%%%%%%%%%%%%
%\function{medfilt1}
%\mpur{Median filter, one dimensional}
%
%\msyn{
%y = medfilt1(x)\\[0pt]
%y = medfilt1(x, w)
%}
%
%\mdes{Perform a one-dimensional median filter of the signal
%\var{x} with a window of width \var{w} (default 5).  End effects
%are handled by replicating the first and last signal values.
%
%Can be 
%}
%
%\vfil\eject 


%%%%%%%%%%%%%%%%%%%%%%%%%%%%%%%%%%%%%%%
\function{mkcube}
\mpur{Create a  cube}

\msyn{
c = mkcube\\[0pt]
c = mkcube(s)\\[0pt]
c = mkcube(s, center)\\[6pt]
c = mkcube2\\[0pt]
c = mkcube2(s)\\[0pt]
c = mkcube2(s, center)
}

\mdes{\var{mkcube} returns an $8\times 3$ matrix where each row
is the coordinates of a vertex of the cube.

\var{mkcube2} returns a $12 \times 6$ matrix where each row
corresponds to one edge of the cube.
The first three elements of each row are the start coordinate of the edge 
and the last three are the end coordinate.

The cube has a side length \var{s} (default 1) and is centered at \var{center}
(default $[0\, 0\, 0]$).
}
\msa{camera}
\vfil\eject 



%%%%%%%%%%%%%%%%%%%%%%%%%%%%%%%%%%%%%%%
\function{mpq, upq, npq}
\mpur{Compute moments of a polygon}

\msyn{
m = mpq(iv, p, q)\\[0pt]
m = upq(iv, p, q)\\[0pt]
m = npq(iv, p, q)
}

\mdes{\var{mpq} computes the pq'th moment of the polygon whose vertices are iv.

\var{upq} and \var{npq} compute the central and normalized-central
moments respectively.}
 
\mcau{Note that the points must be sorted such that they follow the 
perimeter in sequence (either clockwise or anti-clockwise).}

\msa{imoments}
\mref{\nocite{Wilf79}
J.~Wilf and R.~Cunningham, ``Computing region moments from boundary
  representations,'' JPL 79-45, NASA JPL, Nov. 1979.
}
\vfil\eject 

%%%%%%%%%%%%%%%%%%%%%%%%%%%%%%%%%%%%%%%
\function{pnmfilt}
\mpur{Pipe an image through Unix filter}

\msyn{
im2 = pnmfilt(image, cmd)
}

\mdes{Pipes the image through a Unix filter 
program.
The image is written in PGM (P5) format or PPM (P6) format to stdin of
the specified command, and its
output on stdout (assumed
to be PNM format) is returned by this function.
Provides access to many preexisting image program functions that are
part of the PBMplus, ImageMagick and Khoros suites.
}

\mex{To rotate an image we can make use of the 
\texttt{pnmrotate} utility}
\begin{verbatim}
>> lena = loadpgm('lena');
>> rlena = pnmfilt(lena, 'pnmrotate 30');  % rotate by 30 deg
>> image(rlena);
>> colormap(gray(256))
\end{verbatim}

\psfig{figure=figs/pgmfilt.eps,width=8cm}

\mlim{The mechanism is not quick, but it is convenient.  Unfortunately
MATLAB doesn't support proper pipes (could be done with a mex-file...) so
temporary files are used.}

\msa{idisp, xv, savepnm}
\vfil\eject 

%%%%%%%%%%%%%%%%%%%%%%%%%%%%%%%%%%%%%%%
\function{pulnix}
\mpur{Model for Pulnix camera and Digimax digitizer}

\msyn{
cp = pulnix
}

\mdes{Returns the camera calibration matrix for a Pulnix TN-6 camera with
an 8mm lense and a Datacube Digimax digitizer.

The camera parameter object \var{cp} has elements:

\begin{tabular}{lp{10cm}}
\var{cp.f} & focal length (m) \\
\var{cp.u0} & principal point u-coordinate (pix) \\
\var{cp.v0} & principal point v-coordinate (pix) \\
\var{cp.px} & horizontal pixel pitch (pix/m) \\
\var{cp.py} & vertical pixel pitch (pix/m)
\end{tabular}
}

\mex()
\begin{verbatim}
  >> pulnix
  ans = 

     f: 0.0078
    px: -79200
    py: -120500
    u0: 274
    v0: 210
\end{verbatim}

\mref{\nocite{Corke96c}P.~I. Corke, {\em Visual Control of Robots: High-Performance visual servoing}.
\newblock Mechatronics, Research Studies Press (John Wiley), 1996.
}
\msa{camcalp}
\vfil\eject 

%%%%%%%%%%%%%%%%%%%%%%%%%%%%%%%%%%%%%%%%
%\function{rgb2hsi}
%\mpur{RGB color space to hue/saturation/intensity}
%
%\msyn{
%hsi = rgb2hsi(r, g, b)\\[0pt]
%hsi = rgb2hsi(rgb)
%}
%\mdes{Returns a row vector of the HSI values corresponding to 
%the color components \var{r}, \var{g}, and \var{b} which can also
%be given as a 3-element row vector \var{rgb}.  If the components have more
%than one row, the result will be a matrix with one row corresponding to
%each input row.}
%
%\mlim{See also the Matlab builtin function \var{rgb2hsv()}.}
%\mref{An excellent introduction to color spaces can be found at \linebreak
%\texttt{http://www.faqs.org/faqs/graphics/colorspace-faq}}
%\msa{rgb2xyz}
%\vfil\eject 

%%%%%%%%%%%%%%%%%%%%%%%%%%%%%%%%%%%%%%%
\function{rgb2xyz}
\mpur{RGB color space to CIE XYZ}

\msyn{
xyz = rgb2xyz(r, g, b)\\[0pt]
xyz = rgb2xyz(rgb)
}
\mdes{Returns a row vector of the CIE 1931 XYZ values corresponding to 
the color components \var{r}, \var{g}, and \var{b} which can also
be given as a 3-element row vector \var{rgb}.  If the components have more
than one row, the result will be a matrix with one row corresponding to
each input row.}

\msa{rgb2hsv}
\mref{An excellent introduction to color spaces can be found at \linebreak
\texttt{http://www.faqs.org/faqs/graphics/colorspace-faq}}

\vfil\eject 

%%%%%%%%%%%%%%%%%%%%%%%%%%%%%%%%%%%%%%%
\function{rluminos}
\mpur{Relative photopic luminosity}

\msyn{
rluminos(lambda)
}
\mdes{\var{rluminos} returns the relative photopic (light adjusted
cone response) luminosity response of the human eye.
CIE luminosity is obtained by multiplying by 673\unit{lumens/W}.
}

\mex{To show this response over visible wavelengths
}
\begin{verbatim}
>> l = [380:10:700]'*1e-9;        % visible spectrum
>> r = rluminos(l);
>> plot(l*1e9, r)
>> xlabel('Wavelength (nm)')
\end{verbatim}
\noindent
which peaks at around 555\unit{nm}.

\psfig{figure=figs/rlum.eps,width=8cm}

\malg{Evaluated using the Y component of the CIE XYZ color matching
function.}
\msa{cmfxyz}

\vfil\eject 


%%%%%%%%%%%%%%%%%%%%%%%%%%%%%%%%%%%%%%%
\function{saveinr}
\mpur{Save INRIMAGE format image}

\msyn{
saveinr(fname, I)
}

\mdes{\var{saveinr} saves a matrix containing a gray scale
image in an INRIMAGE format file with the 
specified name.  If no extension is
provided an extension of \texttt{.inr} is appended.
This is a binary floating point file format developed at INRIA.
}
\mlim{Only simple 2D images are supported in this implemenation.}

\msa{loadinr}
\vfil\eject 


%%%%%%%%%%%%%%%%%%%%%%%%%%%%%%%%%%%%%%%
\function{savepnm}
\mpur{Save PNM format image}

\msyn{
savepnm(fname, I)\\[0pt]
savepnm(fname, I, comment)
}

\mdes{\var{savepnm} saves a matrix containing an image in binary
greyscale (P5) or RGB color (P6) format to the
file with the specified name.  
The optional comment will be embedded in the image header consistant
with the PBM file format.
}

\msa{loadpgm, loadppm}
\vfil\eject 



%%%%%%%%%%%%%%%%%%%%%%%%%%%%%%%%%%%%%%%
\function{solar}
\mpur{Solar irrandiance spectrum}

\msyn{
p = solar(lambda)
}

\mdes{Return solar irradiance in \unit{W/m^2/nm} for wavelength \var{lambda}.  
\var{lambda} maybe a vector.

}
\mref{{\tt http://www.asdi.com/apps/arm.html}, figure 1.}
\mlim{Solar irrandiance depends on many things: cloud, time, location etc. and this
should be taken as a rough guide only.}

\mex{To show solar irradiance response over visible and
infra-red wavelengths.
}
\begin{verbatim}
>> l = [380:10:1500]'*1e-9;        % visible and IR spectrum
>> s = solar(l);
>> r = rluminos(l);
>> plot(l*1e9, [s r])
>> xlabel('Wavelength (nm)')
>> ylabel('Solar irradiance (W/m^2/nm)')
\end{verbatim}
\noindent
along with the human visible (photopic) response.

\psfig{figure=figs/solar.eps,width=8cm}


\msa{blackbody}
\vfil\eject 

%%%%%%%%%%%%%%%%%%%%%%%%%%%%%%%%%%%%%%%
\function{subpixel}
\mpur{Subpixel interpolation of peak}

\msyn{
[dxr, dyr] = subpixel(surf)\\[0pt]
[dxr, dyr] = subpixel(surf, dx, dy)

}

\mdes{
Given a 2-d surface \var{surf} refine the estimate of the peak to 
subpixel precision using first-order differences.
The peak may be given by \var{(dx, dy)} or searched for.

To find a minimum, call the function with \var{-surf}.

Useful to find peaks in correlation surfaces or Hough accumulator
peaks.

}

\msa{max2d, imatch, ihough}
\vfil\eject 

%%%%%%%%%%%%%%%%%%%%%%%%%%%%%%%%%%%%%%%
\function{testpattern}
\mpur{Create a variety of useful test patterns}

\msyn{
im = testpattern('rampx', w, ncycles)\\[0pt]
im = testpattern('rampy', w, ncycles)\\[6pt]
im = testpattern('sinx', w, ncycles)\\[0pt]
im = testpattern('siny', w, ncycles)\\[6pt]
im = testpattern('dots', w, pitch, diam)\\[6pt]
im = testpattern('squares', w, pitch, s)\\[6pt]
im = testpattern('line', w, theta, c)
}

\mdes{Returns an image of size $w \times w$ containing
a testpattern.  If \var{w} is 2-dimensional it specifies the number of
rows and colums of \var{im}.


With no output arguments the testpattern is displayed using 
\var{idisp()}.

The first four forms create greyscale images with triangular or 
sinusoidal patterns.  For the ramp values are in the range $[0, \, 1]$
and for the sinuoids in the range $[-1, \, 1]$.
If not specified \var{ncycles} corresponds to 1.

The dot and square test patterns are binary images with pixels either
0 or 1.  They are specified in terms of pitch, distance between centres,
and diameter \var{diam} or side length \var{s}.

The line is described by
\[
v = \tan \theta u + c
\]
where $v$ and $u$ are row and column respectively, and \var{theta} is
specified in radians.
Pixels on the line are set to one, elsewhere to zero.
}

\begin{tabular}{cc}
\psfig{figure=figs/tp1.eps,width=5cm} & \psfig{figure=figs/tp2.eps,width=5cm} \\
\psfig{figure=figs/tp3.eps,width=5cm} & \psfig{figure=figs/tp4.eps,width=5cm} \\
\end{tabular}

\mex{}
\begin{verbatim}
>> testpattern('rampx', 256, 2)
>> testpattern('siny', 256, 2)
>> testpattern('dots', 256, 50, 20)
>> testpattern('squares', 256, 64, 16)
\end{verbatim}


\vfil\eject 

%%%%%%%%%%%%%%%%%%%%%%%%%%%%%%%%%%%%%%%
\function{trainseg}
\mpur{Train an rg-space color segmentation table}

\msyn{
map = trainseg(rgb)
}

\mdes{Each pixel of the input color image \var{rgb} is converted to 
normalized $(r,g)$ coordinates 
\begin{eqnarray}
r &=& \frac{R}{R+G+B} \\
g &=& \frac{G}{R+G+B} 
\end{eqnarray}

The function displays a new figure with two windows, the left-hand
is the original color image and the right-hand is the color
segmentation map.
The user clicks on pixels in the left-hand window that belong to the target
set and the corresponding values in rg-space are set in the right-hand image.
The right-hand image is used subsequently for segmentation.

The output is a $256 \times 256$  image with pixel values that are either
0 or 1.  Typically the output image would be further processed with 
morphological closing to create a solid region in rg-space that
represents the range of target colors.
}

\psfig{figure=figs/trainseg.eps,width=8cm}

\mex{Train a color segmentation table of the yellow targets}
\begin{verbatim}
    >> targ = loadppm('target.ppm');
    >> map = trainseg(targ);
\end{verbatim}
Every mouse click in the left-hand window adds a point to the right-hand
window.  By clicking on many points within the target regions we can
build up a generalization of its color, as shown by the finite sized
region in the right-hand window.



\msa{colorseg,imorph} 

\vfil\eject 

%%%%%%%%%%%%%%%%%%%%%%%%%%%%%%%%%%%%%%%
\function{visjac\_p}
\mpur{Visual Jacobian matrix}

\msyn{
J = visjac\_p(uv, z)\\[0pt]
J = visjac\_p(uv, z, f)\\[0pt]
J = visjac\_p(uv, z, cp)\\[0pt]
}


\mdes{Returns a $2 \times 6$ visual motion Jacobian
that maps relative camera motion 
\[
\left[ \begin{array}{c}
	\dot{u} \\ \dot{v} \end{array} \right] = {\bf J}
\left[ \begin{array}{c} T_{x} \cr T_{y} \cr T_{z} \cr \omega_{x}
 \cr \omega_{y} \cr \omega_{z} \end{array} \right]
\]
to image plane velocity for the point \var{uv} = $(u,v)$, 
where the image Jacobian is
\begin{equation}
{\bf J} =\left[  \begin{array}{cccccc}
   {\ds \fl \over \ds z}  &  0  & - {\ds u} \over {\ds z}  &
           - {\ds u v} \over {\ds \fl}  &
           {\ds \fl^2 + u^2 } \over {\ds \fl} &
           -v \\
   0 &  {\ds \fl \over \ds z} & - {\ds v} \over {\ds z}  &
           {\ds - \fl^2 -  v^2 } \over {\ds \fl} &
           {\ds u v} \over {\ds \fl}  &
           u \\
   \end{array} \right] 
\end{equation}

For 3 or more points the Jacobians can be stacked and used to solve for
relative motion given observed image plane motion.

The Jacobian is a function of the camera parameters which can be given as
just a focal length in pixels \var{f}, or as a full camera parameter
object \var{cp}:

\begin{tabular}{lp{10cm}}
\var{cp.f} & focal length (m) \\
\var{cp.u0} & principal point u-coordinate (pix) \\
\var{cp.v0} & principal point v-coordinate (pix) \\
\var{cp.px} & horizontal pixel pitch (pix/m) \\
\var{cp.py} & vertical pixel pitch (pix/m)
\end{tabular}
}

\mex{}
\begin{verbatim}
>> visjac_p([0 0], 2, pulnix)

  ans =

   1.0e+11 *

   -0.0000         0   -0.0001    5.8425   -7.6231   -0.0002
         0   -0.0000   -0.0001    6.8129   -8.8892    0.0003
\end{verbatim}
Which indicates that visual motion will be dominated by $\omega_x$ and
$\omega_y$ camera motion.

\msa{camera, pulnix}
\mref{\nocite{Hutchinson96}
S.~Hutchinson, G.~Hager, and P.~Corke, ``A tutorial on visual servo control,''
  {\em IEEE Transactions on Robotics and Automation}, vol.~12, pp.~651--670,
  Oct. 1996.}
\vfil\eject 

%%%%%%%%%%%%%%%%%%%%%%%%%%%%%%%%%%%%%%%
\function{xv}
\mpur{Display image using XV}

\msyn{
xv(image)
}

\mdes{\var{xv} ships the image off to a background XV process.
XV is a great shareware X program  
for image viewing, manipulation and
format conversion.
This script can be easily edited to use your favourite image browser,
such as \texttt{display}, \texttt{eog}, \texttt{kview} etc.}
}

\mref{XV is available from \texttt{http://www.trilon.com/xv/}}
\msa{pnmfilt}
\vfil\eject 

%%%%%%%%%%%%%%%%%%%%%%%%%%%%%%%%%%%%%%%
%\function{video4linux}
%\mpur{Load an image from a Video4Linux source}
%
%\msyn{
%h = video4linux(device)\\[0pt]
%im = video4linux(h)
%}
%
%\mdes{Video4Linux is a versatile interface to many video sources
%including frame grabbers and firewire sources.
%The first call format is used to open the video source.
%
%Subsequent calls with the second call format return the next image
%from the camera in either grayscale or color format.
%
%}
%
%\mlim{Operates only under Linux and is a mex-file.}
%
%\msa{firewire, webcam}
%\vfil\eject 
%
%%%%%%%%%%%%%%%%%%%%%%%%%%%%%%%%%%%%%%%
\function{webcam}
\mpur{Load an image from a web camera}

\msyn{
im = webcam(url)
}

\mdes{Returns an image from the web camera with the specified URL.
Note that web cameras vary widely in the way they are communicated with.
Some allow for control of many camera parameters such as pan, tilt and zoom
by extra arguments in the URL.
}

\mex{Read an image from a Canon web camera}
\begin{verbatim}
      >> im = webcam('http://10.0.0.80/-wvhttp-01-/GetStillImage?p=5');
      >>
      >> im = webcam('http://www.thesurfclub.com.au/Webcam/image.jpg');
      >> idisp(im);
\end{verbatim}
The first example also sets the pan angle to 5.  The second example
loads an image from a webcam at a beach 100km from my lab!  Not a good day
for the beach today...

\psfig{figure=figs/beach.eps,width=8cm}


\msa{firewire}
\vfil\eject 

%%%%%%%%%%%%%%%%%%%%%%%%%%%%%%%%%%%%%%%
\function{yuvopen}
\mpur{Open a YUV4MPEG format file}

\msyn{
h = yuvopen(file)
}

\mdes{Opens a file organized in YUV4MPEG format.  This is a raw
uncompressed file in 4:2:0 format with YUV color encoding.
Returns a handle to the stream that is used for subsequent read operations.

This file format is used as a precursor to mpeg encoding and can be
played by \texttt{mplayer} and transcoded by \texttt{ffmpeg}.
The stream header is saved in \var{h.hdr}.

See the Berkeley mpeg tools manual for more details.
}

\mlim{Assumes the file is in yuv420 format.}

\msa{yuvread, yuv2rgb}
\vfil\eject 

%%%%%%%%%%%%%%%%%%%%%%%%%%%%%%%%%%%%%%%
\function{yuvread}
\mpur{Read frame from a YUV4MPEG format file}

\msyn{
[y,u,v] = yuvread(h)\\[0pt]
[y,u,v] = yuvread(h, skip)\\[0pt]
[y,u,v,hdr] = yuvread(h, skip)
}

\mdes{Reads the next frame from the YUV4MPEG format file opened on the
handle \var{h}.  The frame is returned as a luminance plane, \var{y}, 
and two half resolution planes \var{u} and \var{v}.
If \var{skip} is provided then \var{skip} (default 0) frames will be 
skipped before the next frame is returned.
The function can optionally return the header string, \var{hdr}, which 
contains information specific to the tool used to create the file.

Returns \var{y = []} on end of file.
}

\mlim{Assumes the file is in yuv420 format.}

\msa{yuvopen, yuv2rgb}
\vfil\eject 

%%%%%%%%%%%%%%%%%%%%%%%%%%%%%%%%%%%%%%%
\function{yuv2rgb}
\mpur{Convert an image from YUV to RGB format}

\msyn{
rgb = yuv2rgb(y, u, v)\\[0pt]
[r,g,b] = yuvread(y, u, v)\\[0pt]
rgb = yuv2rgb2(y, u, v)\\[0pt]
[r,g,b] = yuvread2(y, u, v)
}

\mdes{Converts the YUV image to an RGB image with all planes of the
same size.  The first two calls halve the resolution of luminance, \var{y},
to match \var{u} and \var{v}.  The second two double  \var{u} and \var{v}
using simple pixel replication.
}

\msa{yuvopen, yuvread}
\vfil\eject 

%%%%%%%%%%%%%%%%%%%%%%%%%%%%%%%%%%%%%%%
\function{zcross}
\mpur{Find zero crossings}

\msyn{
zc = zcross(image)
}

\mdes{\var{zcross} returns a binary image where set pixels correspond
to transitions from negative to positive in the input image.
Often used in conjunction with a Laplacian of Gaussian operator which
is the basis of the Marr-Poggio edge finder.
}

\mex{}
\begin{verbatim}
  >> lena = loadpgm('lena');
  >> LoGlena = conv2(lena, klog(1));
  >> idisp(zcross(LoGlena))
\end{verbatim}

\psfig{figure=figs/zclena.eps,width=8cm}
\mlim{The method is quite crude, at each pixel the result is the logical or
of a transition to the left or below.}
\msa{klog}
\vfil\eject 

%%%%%%%%%%%%%%%%%%%%%%%%%%%%%%%%%%%%%%%
\function{zncc}
\mpur{Zero-mean normalized cross-correlation}

\msyn{
m = zncc(A, B)
}

\mdes{
Compute the zero-mean normalized cross-correlation similarity measure
between the two equally sized image patches \var{A} and \var{B}.
\[
ZNCC(A, B)=\frac{\sum (A_{ij}-\overline{A})(B_{ij}-\overline{B})}{\sqrt{\sum (A_{ij
}-\overline{A})\sum (B_{ij}-\overline{B})}}
\]
where $\overline{A}$ and $\overline{B}$ are the mean over the region
being matched. This measure is invariant to illumination offset.
While computationally complex it yields
a well bounded result, simplifying the decision process.
Result is in the range -1 to 1, with
1 indicating identical pixel patterns.
}
\msa{similarity}


\cleardoublepage
\bibliographystyle{ieeetr}
\bibliography{strings,3d,vision,publist}
\end{document}
